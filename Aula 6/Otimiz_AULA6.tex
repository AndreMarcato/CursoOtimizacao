%%%%%%%%%%%%%%%%%%%%%%%%%%%%%%%%%%%%%%%%%
% Beamer Presentation
% LaTeX Template
% Version 1.0 (10/11/12)
%
% This template has been downloaded from:
% http://www.LaTeXTemplates.com
%
% License:
% CC BY-NC-SA 3.0 (http://creativecommons.org/licenses/by-nc-sa/3.0/)
%
%%%%%%%%%%%%%%%%%%%%%%%%%%%%%%%%%%%%%%%%%
%----------------------------------------------------------------------------------------
%	PACKAGES AND THEMES
%----------------------------------------------------------------------------------------
% Interessante: 
%https://www.codecogs.com/latex/eqneditor.php?lang=pt-br

\documentclass{beamer}

\usepackage[brazilian]{babel}
\usepackage[utf8]{inputenc}
\usepackage[T1]{fontenc}
\usepackage{colortbl}
\usepackage{mathrsfs}
\usepackage{smartdiagram}
\usepackage{listings}
\usepackage[framed,numbered,autolinebreaks,useliterate]{mcode}
\usepackage{multirow}
\usepackage{amssymb}
\usepackage{mdframed}
\usepackage{listings}
\usepackage{amsmath}
\usepackage[framed,numbered,autolinebreaks,useliterate]{mcode}
\usepackage{cancel}

\mode<presentation> {

% The Beamer class comes with a number of default slide themes
% which change the colors and layouts of slides. Below this is a list
% of all the themes, uncomment each in turn to see what they look like.

%\usetheme{default}
%\usetheme{AnnArbor}
%\usetheme{Antibes}
%\usetheme{Bergen}
%\usetheme{Berkeley}
%\usetheme{Berlin}
%\usetheme{Boadilla}
%\usetheme{CambridgeUS}
%\usetheme{Copenhagen}
%\usetheme{Darmstadt}
%\usetheme{Dresden}
\usetheme{Frankfurt}
%\usetheme{Goettingen}
%\usetheme{Hannover}
%\usetheme{Ilmenau}
%\usetheme{JuanLesPins}
%\usetheme{Luebeck}
%\usetheme{Madrid}
%\usetheme{Malmoe}
%\usetheme{Marburg}
%\usetheme{Montpellier}
%\usetheme{PaloAlto}
%\usetheme{Pittsburgh}
%\usetheme{Rochester}
%\usetheme{Singapore}
%\usetheme{Szeged}
%\usetheme{Warsaw}

% As well as themes, the Beamer class has a number of color themes
% for any slide theme. Uncomment each of these in turn to see how it
% changes the colors of your current slide theme.

%\usecolortheme{albatross}
%\usecolortheme{beaver}
%\usecolortheme{beetle}
%\usecolortheme{crane}
%\usecolortheme{dolphin}
%\usecolortheme{dove}
%\usecolortheme{fly}
%\usecolortheme{lily}
%\usecolortheme{orchid}
%\usecolortheme{rose}
%\usecolortheme{seagull}
%\usecolortheme{seahorse}
%\usecolortheme{whale}
%\usecolortheme{wolverine}

%\setbeamertemplate{footline} % To remove the footer line in all slides uncomment this line
%\setbeamertemplate{footline}[page number] % To replace the footer line in all slides with a simple slide count uncomment this line

%\setbeamertemplate{navigation symbols}{} % To remove the navigation symbols from the bottom of all slides uncomment this line
}

\usepackage{graphicx} % Allows including images
\usepackage{booktabs} % Allows the use of \toprule, \midrule and \bottomrule in tables

\usepackage{animate}
\usepackage{hyperref}
\usepackage{media9}
\usepackage{listings}
\usepackage{amsmath}
\usepackage[framed,numbered,autolinebreaks,useliterate]{mcode}
\usepackage{chronology}
\usepackage{xpatch}
\xpretocmd{\chronology}{\tikzset{>=|}}{}{\failure}
\usepackage{mdframed}
\usepackage{tikz}
\usepackage{makecell}
\usepackage{array}

\usepackage{tikz}
\usetikzlibrary{shapes.geometric}
\usetikzlibrary{shapes,arrows}
\usepackage{array}

%----------------------------------------------------------------------------------------
%	TITLE PAGE
%----------------------------------------------------------------------------------------

\logo
{
    \includegraphics[width=0.6cm,height=0.6cm,keepaspectratio]{UFJF.jpg}~%
}

\title[Aula 6]{Análise de Sensibilidade} 

\author{\scriptsize Professores André L.M. Marcato, Ivo C.da Silva Jr, João A.Passos Filho } % Your name
\institute[UFJF/PPEE]{Universidade Federal de Juiz de Fora \\
	Programa de Pós-Graduação em Engenharia Elétrica \\
	\medskip
	\textit{\href{mailto:andre.marcato@ufjf.edu.br}{andre.marcato@ufjf.edu.br}, \href{mailto:ivo.chaves@ufjf.edu.br}{ivo.junior@ufjf.edu.br}, \href{mailto:joao.passos@ufjf.edu.br}{joao.passos@ufjf.edu.br}}
}

%\date{\small \today} % Date, can be changed to a custom date
\date{\small Primeiro Semestre de 2018} % Date, can be changed to a custom date

\hypersetup{
    colorlinks=true,
    linkcolor=gray,
    filecolor=magenta,      
    urlcolor=cyan,
}

\begin{document}

\begin{frame}
	\titlepage 
	\begin{figure}[!htb]
		\centering
		\includegraphics[width=2.6cm, height=1.7cm]{cover.jpg}
	\end{figure}
\end{frame}

\begin{frame}
	\frametitle{Agenda da Apresentação}
	\tableofcontents 
\end{frame}

%----------------------------------------------------------------------------------------
%	PRESENTATION SLIDES
%----------------------------------------------------------------------------------------

%\section{Método do Big-M}
\subsection{Restrições de Igualdade / Maior ou Igual }
\begin{frame}
	\frametitle{Restrições de Igualdade / Maior ou Igual}
	\begin{columns}
		\begin{column}{0.4\textwidth}
			\includegraphics[width=5cm,height=4cm]{fique-de-olho.jpg}
		\end{column}
		\begin{column}{0.5\textwidth}
			\begin{itemize}
			\item Soluções Básicas Iniciais devem ser factíveis.
			\item Soluções Factíveis são diretas quando todas as restrições do modelo são da forma $\le$.
			\item Restrições do modelo na forma $\ge$ ou $=$ não levam a obenção direta de soluções iniciais factíveis.
			\end{itemize}
		\end{column}
	\end{columns}
\end{frame}

\begin{frame}
	\frametitle{Exemplo}
	\begin{columns}
		\begin{column}{0.5\textwidth}
			\begin{block}{Encontre uma SBF Inicial para:}
				\centering
				$
					\begin{matrix}
						\scriptstyle \max Z = 3x_1 + 2,5x_2 + 1,2x_3 \\
						\scriptstyle \text{Sujeito à:}\\
						\scriptstyle x_1 -2x_2 + 4x_3 \le 40 \\
						\scriptstyle x_1 +x_2  + 2x_3 \ge 60 \\
						\scriptstyle 2x_1 +3x_2 + x_3 = 15 \\
						\scriptstyle x_1, x_2, x_3 \ge 0 \\
					\end{matrix}
				$
			\end{block}
		\end{column} \pause
		\begin{column}{0.5\textwidth}
			\begin{exampleblock}{Forma Padrão:}
				\centering
				$
					\begin{matrix}
						\scriptstyle \max Z = 3x_1 + 2,5x_2 + 1,2x_3 \\
						\scriptstyle \text{Sujeito à:}\\
						\scriptstyle x_1 -2x_2 + 4x_3 {\color{red}+x_4 =} 40 \\
						\scriptstyle x_1 +x_2  + 2x_3 {\color{red}-x_5=} 60 \\
						\scriptstyle 2x_1 +3x_2 + x_3 = 15 \\
						\scriptstyle x_1, x_2, x_3{\color{red},x_4,x_5} \ge 0 \\
					\end{matrix}
				$
			\end{exampleblock}
		\end{column} \pause
	\end{columns}

	\vspace{0.8cm}

	\begin{columns}
		\begin{column}{0.5\textwidth}
			$
				\begin{matrix}
					VNB \left\{
								\begin{matrix}
												x_1 = 0 \\
												x_2 = 0 \\
												x_3 = 0 \\
								\end{matrix} 
						\right. &
					VB \left\{
								\begin{matrix}
												x_4 = 40 \\
												{\color{red} x_5 = -60} \\
								\end{matrix} 
						\right. \\		
					 &  \\				
				\end{matrix}
			$
			$ {\color{red}0 + 0 + 0 = 15}  $
			
		\end{column} \pause
		\begin{column}{0.5\textwidth}
			\centering
			\includegraphics[width=3cm,height=3cm]{houston.jpg}
		\end{column}
	\end{columns}
\end{frame}

\begin{frame}
	\centering
	\includegraphics[width=10cm,height=3cm]{Pergunta_Final.png}
	\begin{columns}
		\begin{column}{0.33\textwidth}
			\only<1>
			{
				\begin{mdframed}[backgroundcolor=green!80]
					\centering
					Método das Penalidades (BIG-M)
				\end{mdframed}
			}		
			\only<2>
			{
				\begin{mdframed}[backgroundcolor=red!80]
					\centering
					Método das Penalidades (BIG-M)
				\end{mdframed}
			}		
		\end{column}
		\begin{column}{0.33\textwidth}
			\centering
			\includegraphics[width=4cm,height=4cm]{twoways.jpg}
		\end{column}
		\hspace{0.2cm}
		\begin{column}{0.33\textwidth}
			\begin{mdframed}[backgroundcolor=green!80]
				\centering
				Método das Duas Fases
			\end{mdframed}		
		\end{column}
	\end{columns}
\end{frame}

\begin{frame}
	\frametitle{O Método do Big M}
	\begin{itemize}
	\item[] {Este método consiste em acrescentar na FOB do problema original (\textbf{forma padrão}) variáveis artificiais - $\mathbf{\color{red}a}$ - com penalidades - $\mathbf{\color{red}M}$:} \pause
		\begin{itemize}
		\item {Negativos Muito Grandes - Problemas de Maximização} 
			\begin{mdframed}[backgroundcolor=green!80,rightmargin=4.8cm]
			$
			\max Z = 2x_1 + 3x_2 \color{red}\boxed{-Ma}
			$
			\end{mdframed} \pause
		\item {Positivos muito grandes - Problemas de Minimização}
			\begin{mdframed}[backgroundcolor=green!80,rightmargin=4.8cm]
			$
			\max Z = 2x_1 + 3x_2 \color{red}\boxed{+Ma}
			$
			\end{mdframed} \pause
		\end{itemize}
	\end{itemize}
	\begin{columns}
		\begin{column}{0.3\textwidth}
			\centering
			\includegraphics[width=3cm,height=3cm]{importante.jpg} 
		\end{column}
		\begin{column}{0.7\textwidth}
			Na solução final os valores das variáveis artificiais devem ser nulos (VNB).		
		\end{column}
	\end{columns}
\end{frame}

\subsection{Exemplo}

\begin{frame}
	\frametitle{Exemplo}
	\begin{columns}
		\begin{column}{5cm}
			\begin{block}{Resolver o seguinte PPL}
				\centering
				$
					\begin{matrix}
						\max 2x_1+3x_2 \\
						\text{Sujeito a} \\
						-2x_1+3x_2 \le 6 \\
						x_1 + 2x_2 \ge 8 \\
						x_1 + x_2 = 6 \\
						x_1, x_2 \ge 0 \\
					\end{matrix}
				$
			\end{block} \pause
		\end{column}
		\begin{column}{5cm}
			\begin{exampleblock}{Forma Padrão}
				\centering
				$
					\begin{matrix}
						\max 2x_1+3x_2 \\
						\text{Sujeito a} \\
						-2x_1+3x_2 {\color{red}+x_3=} 6 \\
						x_1 + 2x_2 {\color{red}-x_4=} 8 \\
						x_1 + x_2 = 6 \\
						x_1, x_2{\color{red},x_3,x_4} \ge 0 \\
					\end{matrix}
				$
			\end{exampleblock} \pause
		\end{column}
	\end{columns}
	\vspace{0.5cm}
	\begin{columns}
		\begin{column}{4cm}
			$
				\begin{matrix}
					VNB \left\{ 
								\begin{matrix}
												x_1 = 0 \\
												x_2 = 0 \\
								\end{matrix}
						\right. &
					VB \left\{
								\begin{matrix}
									x_3 = 6 \\
									x_4 = -8 \\
								\end{matrix}
						\right. \\ 
					& \\
				\end{matrix}
			$
			$ Z = 0 $ \pause
		\end{column}
		\begin{column}{5cm}
			\scriptsize
			\begin{itemize}
			\item[] \underline{Problemas:}
			\item $x_4 = 8$, deveria ser maior que zero.
			\item $x_1+x_2=6$, igualdade não satisfeita.
			\item Não possui solução inicial trivial.
			\end{itemize}
		\end{column}
	\end{columns}
\end{frame}


\begin{frame}
	\frametitle{Exemplo}
	\begin{columns}
		\begin{column}{5cm}
			\begin{block}{Resolver o seguinte PPL}
				\centering
				$
					\begin{matrix}
						\max 2x_1+3x_2 \\
						\text{Sujeito a} \\
						-2x_1+3x_2 \le 6 \\
						x_1 + 2x_2 \ge 8 \\
						x_1 + x_2 = 6 \\
						x_1, x_2 \ge 0 \\
					\end{matrix}
				$
			\end{block}
		\end{column}
		\begin{column}{5cm}
			\begin{exampleblock}{Forma Padrão}
				\centering
				$
					\begin{matrix}
						\max 2x_1+3x_2 \\
						\text{Sujeito a} \\
						-2x_1+3x_2 {\color{red}+x_3=} 6 \\
						x_1 + 2x_2 {\color{red}-x_4=} 8 \\
						x_1 + x_2 = 6 \\
						x_1, x_2{\color{red},x_3,x_4} \ge 0 \\
					\end{matrix}
				$
			\end{exampleblock}
		\end{column}
	\end{columns}
	\only<1>
	{
		\begin{alertblock}{\centering Inserção das Variáveis Artificiais e das Penalidades}
					\centering
					$
						\begin{matrix}
							\max Z - 2x_1-3x_2  &                                       &                   & \cellcolor{green!70}{\color{red}+Mx_5} & \cellcolor{green!70}{\color{red}+ Mx_6}  & =0 \\
							\text{Sujeito a} \\
							-2x_1+3x_2          & \cellcolor{green!70}{\color{red}+x_3} &                   &                                        &                                          & =6 \\
							x_1 + 2x_2          &                                       & {\color{red}-x_4} & \cellcolor{green!70}{\color{red}+x_5}  &                                          & =8 \\
							x_1 + x_2           &                                       &                   &                                        & \cellcolor{green!70}{\color{red}+x_6}    & =6 \\
							x_1, x_2{\color{red},x_3,x_4,x_5,x_6} \ge 0 \\
						\end{matrix}
					$
		\end{alertblock}
	}
	\only<2-3>
	{
			\vspace{0.5cm}
			\centering
			$
				\begin{matrix}
					VNB \left\{ 
								\begin{matrix}
												x_1 = 0 \\
												x_2 = 0 \\
												x_4 = 0 \\
								\end{matrix}
						\right. &
					VB \left\{
								\begin{matrix}
									x_3 = 6 \\
									x_5 = 8 \\
									x_6 = 6 \\
								\end{matrix}
						\right. 
					& Z = 0\\
				\end{matrix}
			$
	}	
	\only<3>
	{
		\begin{columns}
			\begin{column}{1cm}
				\includegraphics[width=2cm,height=1.3cm]{atencao.jpg}
			\end{column}
			\begin{column}{9cm}
				\color{red}\Large A FOB deve conter somente VNB !
			\end{column}
		\end{columns}
	}
\end{frame}

\begin{frame}
	\frametitle{Exemplo}
	\only<1-2>
	{
		\begin{alertblock}{\centering Inserção das Variáveis Artificiais e das Penalidades}
					\centering
					$
						\begin{matrix}
							\max Z - 2x_1-3x_2  &                                       &                   & \cellcolor{green!70}{\color{red}+Mx_5} & \cellcolor{green!70}{\color{red}+ Mx_6}  & =0 \\
							\text{Sujeito a} \\
							-2x_1+3x_2          & \cellcolor{green!70}{\color{red}+x_3} &                   &                                        &                                          & =6 \\
							x_1 + 2x_2          &                                       & {\color{red}-x_4} & \cellcolor{green!70}{\color{red}+x_5}  &                                          & =8 \\
							x_1 + x_2           &                                       &                   &                                        & \cellcolor{green!70}{\color{red}+x_6}    & =6 \\
							x_1, x_2{\color{red},x_3,x_4,x_5,x_6} \ge 0 \\
						\end{matrix}
					$
		\end{alertblock}
	}	
	\only<3>
	{
		\begin{alertblock}{\centering Inserção das Variáveis Artificiais e das Penalidades}
					\centering
					$
						\begin{matrix}
							\scriptstyle \max Z - 2x_1-3x_2  &                                                    &                               & \scriptstyle \cellcolor{green!70}{\color{red}+M(8-x_1-2x_2+x_4)} & \scriptstyle \cellcolor{green!70}{\color{red}+ M(6-x_1-x_2)}  & \scriptstyle =0 \\
							\scriptstyle \text{Sujeito a} \\
							\scriptstyle -2x_1+3x_2          & \scriptstyle \cellcolor{green!70}{\color{red}+x_3} &                               &                                                                  &                                                               & \scriptstyle =6 \\
							\scriptstyle x_1 + 2x_2          &                                                    & \scriptstyle{\color{red}-x_4} & \scriptstyle \cellcolor{green!70}{\color{red}+x_5}               &                                                               & \scriptstyle =8 \\
							\scriptstyle x_1 + x_2           &                                                    &                               &                                                                  & \scriptstyle \cellcolor{green!70}{\color{red}+x_6}            & \scriptstyle =6 \\
							\scriptstyle x_1, x_2{\color{red},x_3,x_4,x_5,x_6} \ge 0 \\
						\end{matrix}
					$
		\end{alertblock}
	}
	\only<4-5>
	{
		\begin{alertblock}{\centering Inserção das Variáveis Artificiais e das Penalidades}

			$ \max Z + (-2-2M) x_1 + (-3-3M) x_2 + Mx_4 + 14M = 0 $ \\
			{\textbf{Sujeito à}}: \\
			$
				\begin{matrix}
					-2x_1+3x_2+   & {\color{red}x_3} &                  &  					& = 6 \\
					x_1+2x_2-x_4  &				     & \color{red}+x_5  &  					& = 8 \\
					x_1+x_2		  &			         &				   & \color{red}+x_6 	& = 6 \\
					x_1, x_2, x_3,x_4,x_5,x_6 \ge 0  \\
				\end{matrix}
			$
		\end{alertblock}
	}		
	
	\only<2-3>
	{
		\begin{itemize}
		\item[] $x_5 = 8-x_1-2x_2+x_4$
		\item[] $x_6 = 6-x_1-x_2$
		\end{itemize}
	
	}
	\only<4-5>
	{
			\vspace{0.5cm}
			\centering
			$
				\begin{matrix}
					VNB \left\{ 
								\begin{matrix}
												x_1 = 0 \\
												x_2 = 0 \\
												x_4 = 0 \\
								\end{matrix}
						\right. &
					VB \left\{
								\begin{matrix}
									x_3 = 6 \\
									x_5 = 8 \\
									x_6 = 6 \\
								\end{matrix}
						\right. 
					& Z = -14M\\
				\end{matrix}
			$
	}
	\only<5>
	{
		\begin{columns}
			\begin{column}{1cm}
				\includegraphics[width=2cm,height=2cm]{OK.png}
			\end{column}
			\begin{column}{9cm}
				\color{red}\Large A FOB contém somente VNB ! \\
				\color{black}\normalsize \underline{Em geral}, atribui-se a M um valor 20 vezes superior ao maior coeficiente da FOB original.
			\end{column}
		\end{columns}
	}	
\end{frame}

\begin{frame}
	\only<1-5>
	{
		\frametitle{Exemplo - Busca Tableau - \color{cyan} $1^a$ Iteração}
	}
	\only<6-9>
	{
		\frametitle{Exemplo - Busca Tableau - \color{cyan} $2^a$ Iteração}
	}
	\only<10-13>
	{
		\frametitle{Exemplo - Busca Tableau - \color{cyan} $3^a$ Iteração}
	}
	\only<14->
	{
		\frametitle{Exemplo - Busca Tableau - \color{cyan} $4^a$ Iteração}
	}

	\only<1>
	{
		\begin{table}		
			\begin{tabular}{c c c c c c c c c c c c}
				\cline{1-8} 
				\cellcolor{blue!100} \color{white} \scriptsize Base 
				&\cellcolor{blue!100} \color{white} \scriptsize Z 
				&\cellcolor{blue!100} \color{white} $\scriptstyle X_1$ 
				&\cellcolor{blue!100} \color{white} $\scriptstyle X_2$ 
				&\cellcolor{blue!100} \color{red}   $\scriptstyle X_3$ 
				&\cellcolor{blue!100} \color{white} $\scriptstyle X_4$ 
				&\cellcolor{blue!100} \color{red}   $\scriptstyle X_5$ 
				&\cellcolor{blue!100} \color{red}   $\scriptstyle X_6$ 
				&\cellcolor{blue!100} \color{white} \scriptsize b
				&
				&
				& \\
				\cellcolor{blue!100} \color{red} $\scriptstyle X_3$
				& \cellcolor{yellow!50} $\scriptstyle 0$
				& \cellcolor{yellow!50} $\scriptstyle -2$
				& \cellcolor{yellow!50} $\scriptstyle +3$
				& \cellcolor{yellow!50} $\scriptstyle +1$
				& \cellcolor{yellow!50} $\scriptstyle 0$
				& \cellcolor{yellow!50} $\scriptstyle 0$
				& \cellcolor{yellow!50} $\scriptstyle 0$
				& \cellcolor{yellow!50} $\scriptstyle +6$ \\
			    \cellcolor{blue!100} \color{red} $\scriptstyle X_5$
				& \cellcolor{yellow!50} $\scriptstyle 0$
				& \cellcolor{yellow!50} $\scriptstyle +1$
				& \cellcolor{yellow!50} $\scriptstyle +2$
				& \cellcolor{yellow!50} $\scriptstyle 0$			
				& \cellcolor{yellow!50} $\scriptstyle -1$
				& \cellcolor{yellow!50} $\scriptstyle +1$
				& \cellcolor{yellow!50} $\scriptstyle 0$ 
				& \cellcolor{yellow!50} $\scriptstyle +8$ \\
				\cellcolor{blue!100} \color{red} $\scriptstyle X_6$
				& \cellcolor{yellow!50} $\scriptstyle 0$
				& \cellcolor{yellow!50} $\scriptstyle +1$
				& \cellcolor{yellow!50} $\scriptstyle +1$
				& \cellcolor{yellow!50} $\scriptstyle 0$
				& \cellcolor{yellow!50} $\scriptstyle 0$
				& \cellcolor{yellow!50} $\scriptstyle 0$
				& \cellcolor{yellow!50} $\scriptstyle +1$
				& \cellcolor{yellow!50} $\scriptstyle +6$ \\
				\cellcolor{blue!100} \color{white} $\scriptstyle Z$
				& \cellcolor{yellow!50} $\scriptstyle +1$
				& \cellcolor{yellow!50} $\scriptstyle (-2-2M)$
				& \cellcolor{yellow!50} $\scriptstyle (-3-3M)$
				& \cellcolor{yellow!50} $\scriptstyle 0$
				& \cellcolor{yellow!50} $\scriptstyle M$
				& \cellcolor{yellow!50} $\scriptstyle 0$
				& \cellcolor{yellow!50} $\scriptstyle 0$ 
				& \cellcolor{yellow!50} $\scriptstyle -14M$  \\
			\end{tabular}
		\end{table}			
	}
	\only<2>
	{
		\begin{table}		
			\begin{tabular}{c c c c c c c c c c c c}
				\cline{1-8} 
				\cellcolor{blue!100} \color{white} \scriptsize Base 
				&\cellcolor{blue!100} \color{white} \scriptsize Z 
				&\cellcolor{blue!100} \color{white} $\scriptstyle X_1$ 
				&\cellcolor{blue!100} \color{white} $\scriptstyle X_2$ 
				&\cellcolor{blue!100} \color{red}   $\scriptstyle X_3$ 
				&\cellcolor{blue!100} \color{white} $\scriptstyle X_4$ 
				&\cellcolor{blue!100} \color{red}   $\scriptstyle X_5$ 
				&\cellcolor{blue!100} \color{red}   $\scriptstyle X_6$ 
				&\cellcolor{blue!100} \color{white} \scriptsize b
				&
				&
				& \\
				\cellcolor{blue!100} \color{red} $\scriptstyle X_3$
				& \cellcolor{yellow!50} $\scriptstyle 0$
				& \cellcolor{yellow!50} $\scriptstyle -2$
				& \cellcolor{yellow!50} $\scriptstyle +3$
				& \cellcolor{yellow!50} $\scriptstyle +1$
				& \cellcolor{yellow!50} $\scriptstyle 0$
				& \cellcolor{yellow!50} $\scriptstyle 0$
				& \cellcolor{yellow!50} $\scriptstyle 0$
				& \cellcolor{yellow!50} $\scriptstyle +6$ \\
			    \cellcolor{blue!100} \color{red} $\scriptstyle X_5$
				& \cellcolor{yellow!50} $\scriptstyle 0$
				& \cellcolor{yellow!50} $\scriptstyle +1$
				& \cellcolor{yellow!50} $\scriptstyle +2$
				& \cellcolor{yellow!50} $\scriptstyle 0$			
				& \cellcolor{yellow!50} $\scriptstyle -1$
				& \cellcolor{yellow!50} $\scriptstyle +1$
				& \cellcolor{yellow!50} $\scriptstyle 0$ 
				& \cellcolor{yellow!50} $\scriptstyle +8$ \\
				\cellcolor{blue!100} \color{red} $\scriptstyle X_6$
				& \cellcolor{yellow!50} $\scriptstyle 0$
				& \cellcolor{yellow!50} $\scriptstyle +1$
				& \cellcolor{yellow!50} $\scriptstyle +1$
				& \cellcolor{yellow!50} $\scriptstyle 0$
				& \cellcolor{yellow!50} $\scriptstyle 0$
				& \cellcolor{yellow!50} $\scriptstyle 0$
				& \cellcolor{yellow!50} $\scriptstyle +1$
				& \cellcolor{yellow!50} $\scriptstyle +6$ \\
				\cellcolor{blue!100} \color{white} $\scriptstyle Z$
				& \cellcolor{yellow!50} $\scriptstyle +1$
				& \cellcolor{yellow!50} $\scriptstyle -122$
				& \cellcolor{yellow!50} $\scriptstyle -183$
				& \cellcolor{yellow!50} $\scriptstyle 0$
				& \cellcolor{yellow!50} $\scriptstyle +60$
				& \cellcolor{yellow!50} $\scriptstyle 0$
				& \cellcolor{yellow!50} $\scriptstyle 0$ 
				& \cellcolor{yellow!50} $\scriptstyle -840$  \\
			\end{tabular}
		\end{table}			
	}
	\only<3>
	{
		\begin{table}		
			\begin{tabular}{c c c c c c c c c c c c}
				\cline{1-8} 
				\cellcolor{blue!100} \color{white} \scriptsize Base 
				&\cellcolor{blue!100} \color{white} \scriptsize Z 
				&\cellcolor{blue!100} \color{white} $\scriptstyle X_1$ 
				&\cellcolor{blue!100} \color{white} $\scriptstyle X_2$ 
				&\cellcolor{blue!100} \color{red}   $\scriptstyle X_3$ 
				&\cellcolor{blue!100} \color{white} $\scriptstyle X_4$ 
				&\cellcolor{blue!100} \color{red}   $\scriptstyle X_5$ 
				&\cellcolor{blue!100} \color{red}   $\scriptstyle X_6$ 
				&\cellcolor{blue!100} \color{white} \scriptsize b
				&
				&
				& \\
				\cellcolor{blue!100} \color{red} $\scriptstyle X_3$
				& \cellcolor{yellow!50} $\scriptstyle 0$
				& \cellcolor{yellow!50} $\scriptstyle -2$
				& \cellcolor{gray!50} $\scriptstyle +3$
				& \cellcolor{yellow!50} $\scriptstyle +1$
				& \cellcolor{yellow!50} $\scriptstyle 0$
				& \cellcolor{yellow!50} $\scriptstyle 0$
				& \cellcolor{yellow!50} $\scriptstyle 0$
				& \cellcolor{yellow!50} $\scriptstyle +6$ \\
			    \cellcolor{blue!100} \color{red} $\scriptstyle X_5$
				& \cellcolor{yellow!50} $\scriptstyle 0$
				& \cellcolor{yellow!50} $\scriptstyle +1$
				& \cellcolor{gray!50} $\scriptstyle +2$
				& \cellcolor{yellow!50} $\scriptstyle 0$			
				& \cellcolor{yellow!50} $\scriptstyle -1$
				& \cellcolor{yellow!50} $\scriptstyle +1$
				& \cellcolor{yellow!50} $\scriptstyle 0$ 
				& \cellcolor{yellow!50} $\scriptstyle +8$ \\
				\cellcolor{blue!100} \color{red} $\scriptstyle X_6$
				& \cellcolor{yellow!50} $\scriptstyle 0$
				& \cellcolor{yellow!50} $\scriptstyle +1$
				& \cellcolor{gray!50} $\scriptstyle +1$
				& \cellcolor{yellow!50} $\scriptstyle 0$
				& \cellcolor{yellow!50} $\scriptstyle 0$
				& \cellcolor{yellow!50} $\scriptstyle 0$
				& \cellcolor{yellow!50} $\scriptstyle +1$
				& \cellcolor{yellow!50} $\scriptstyle +6$ \\
				\cellcolor{blue!100} \color{white} $\scriptstyle Z$
				& \cellcolor{yellow!50} $\scriptstyle +1$
				& \cellcolor{yellow!50} $\scriptstyle -122$
				& \cellcolor{gray!50} $\scriptstyle -183$
				& \cellcolor{yellow!50} $\scriptstyle 0$
				& \cellcolor{yellow!50} $\scriptstyle +60$
				& \cellcolor{yellow!50} $\scriptstyle 0$
				& \cellcolor{yellow!50} $\scriptstyle 0$ 
				& \cellcolor{yellow!50} $\scriptstyle -840$  \\
				& & & \includegraphics[width=0.3cm,height=0.3cm]{setacima.jpg} \\
				& & & \scriptsize Entra\\ 
			\end{tabular}
		\end{table}			
	}
	\only<4>
	{
		\begin{table}		
			\begin{tabular}{c c c c c c c c c c c c}
				\cline{1-8} 
				\cellcolor{blue!100} \color{white} \scriptsize Base 
				&\cellcolor{blue!100} \color{white} \scriptsize Z 
				&\cellcolor{blue!100} \color{white} $\scriptstyle X_1$ 
				&\cellcolor{blue!100} \color{white} $\scriptstyle X_2$ 
				&\cellcolor{blue!100} \color{red}   $\scriptstyle X_3$ 
				&\cellcolor{blue!100} \color{white} $\scriptstyle X_4$ 
				&\cellcolor{blue!100} \color{red}   $\scriptstyle X_5$ 
				&\cellcolor{blue!100} \color{red}   $\scriptstyle X_6$ 
				&\cellcolor{blue!100} \color{white} \scriptsize b
				&
				&
				& \\
				\cellcolor{blue!100} \color{red} $\scriptstyle X_3$
				& \cellcolor{yellow!50} $\scriptstyle 0$
				& \cellcolor{yellow!50} $\scriptstyle -2$
				& \cellcolor{gray!50} $\scriptstyle +3$
				& \cellcolor{yellow!50} $\scriptstyle +1$
				& \cellcolor{yellow!50} $\scriptstyle 0$
				& \cellcolor{yellow!50} $\scriptstyle 0$
				& \cellcolor{yellow!50} $\scriptstyle 0$
				& \cellcolor{gray!50} $\scriptstyle +6$
				& $\scriptstyle \frac{+6}{+3}=+2$ & \includegraphics[width=0.3cm,height=0.3cm]{setaesquerda.jpg} \\
			    \cellcolor{blue!100} \color{red} $\scriptstyle X_5$
				& \cellcolor{yellow!50} $\scriptstyle 0$
				& \cellcolor{yellow!50} $\scriptstyle +1$
				& \cellcolor{gray!50} $\scriptstyle +2$
				& \cellcolor{yellow!50} $\scriptstyle 0$			
				& \cellcolor{yellow!50} $\scriptstyle -1$
				& \cellcolor{yellow!50} $\scriptstyle +1$
				& \cellcolor{yellow!50} $\scriptstyle 0$ 
				& \cellcolor{gray!50} $\scriptstyle +8$ 
				& $\scriptstyle \frac{+8}{+2}=+4$ & \includegraphics[width=0.3cm,height=0.3cm]{setaesquerda.jpg} \\
				\cellcolor{blue!100} \color{red} $\scriptstyle X_6$
				& \cellcolor{yellow!50} $\scriptstyle 0$
				& \cellcolor{yellow!50} $\scriptstyle +1$
				& \cellcolor{gray!50} $\scriptstyle +1$
				& \cellcolor{yellow!50} $\scriptstyle 0$
				& \cellcolor{yellow!50} $\scriptstyle 0$
				& \cellcolor{yellow!50} $\scriptstyle 0$
				& \cellcolor{yellow!50} $\scriptstyle +1$
				& \cellcolor{gray!50} $\scriptstyle +6$ 
				& $\scriptstyle \frac{+6}{+1}=+6$ & \includegraphics[width=0.3cm,height=0.3cm]{setaesquerda.jpg}\\
				\cellcolor{blue!100} \color{white} $\scriptstyle Z$
				& \cellcolor{yellow!50} $\scriptstyle +1$
				& \cellcolor{yellow!50} $\scriptstyle -122$
				& \cellcolor{gray!50} $\scriptstyle -183$
				& \cellcolor{yellow!50} $\scriptstyle 0$
				& \cellcolor{yellow!50} $\scriptstyle +60$
				& \cellcolor{yellow!50} $\scriptstyle 0$
				& \cellcolor{yellow!50} $\scriptstyle 0$ 
				& \cellcolor{gray!50} $\scriptstyle -840$  \\
				& & & \includegraphics[width=0.3cm,height=0.3cm]{setacima.jpg} \\
				& & & \scriptsize Entra \\ 
			\end{tabular}
		\end{table}			
	}
	\only<5>
	{
		\begin{table}		
			\begin{tabular}{c c c c c c c c c c c c}
				\cline{1-8} 
				\cellcolor{blue!100} \color{white} \scriptsize Base 
				&\cellcolor{blue!100} \color{white} \scriptsize Z 
				&\cellcolor{blue!100} \color{white} $\scriptstyle X_1$ 
				&\cellcolor{blue!100} \color{white} $\scriptstyle X_2$ 
				&\cellcolor{blue!100} \color{red}   $\scriptstyle X_3$ 
				&\cellcolor{blue!100} \color{white} $\scriptstyle X_4$ 
				&\cellcolor{blue!100} \color{red}   $\scriptstyle X_5$ 
				&\cellcolor{blue!100} \color{red}   $\scriptstyle X_6$ 
				&\cellcolor{blue!100} \color{white} \scriptsize b
				&
				&
				& \\
				\cellcolor{blue!100} \color{red} $\scriptstyle X_3$
				& \cellcolor{gray!50} $\scriptstyle 0$
				& \cellcolor{gray!50} $\scriptstyle -2$
				& \cellcolor{red!50} $\scriptstyle +3$
				& \cellcolor{gray!50} $\scriptstyle +1$
				& \cellcolor{gray!50} $\scriptstyle 0$
				& \cellcolor{gray!50} $\scriptstyle 0$
				& \cellcolor{gray!50} $\scriptstyle 0$
				& \cellcolor{gray!50} $\scriptstyle +6$
				& $\scriptstyle \frac{+6}{+3}=+2$ &
				\includegraphics[width=0.3cm,height=0.3cm]{setaesquerda.jpg} \scriptsize Sai \\
			    \cellcolor{blue!100} \color{red} $\scriptstyle X_5$
				& \cellcolor{yellow!50} $\scriptstyle 0$
				& \cellcolor{yellow!50} $\scriptstyle +1$
				& \cellcolor{gray!50} $\scriptstyle +2$
				& \cellcolor{yellow!50} $\scriptstyle 0$			
				& \cellcolor{yellow!50} $\scriptstyle -1$
				& \cellcolor{yellow!50} $\scriptstyle +1$
				& \cellcolor{yellow!50} $\scriptstyle 0$ 
				& \cellcolor{gray!50} $\scriptstyle +8$  \\
				\cellcolor{blue!100} \color{red} $\scriptstyle X_6$
				& \cellcolor{yellow!50} $\scriptstyle 0$
				& \cellcolor{yellow!50} $\scriptstyle +1$
				& \cellcolor{gray!50} $\scriptstyle +1$
				& \cellcolor{yellow!50} $\scriptstyle 0$
				& \cellcolor{yellow!50} $\scriptstyle 0$
				& \cellcolor{yellow!50} $\scriptstyle 0$
				& \cellcolor{yellow!50} $\scriptstyle +1$
				& \cellcolor{gray!50} $\scriptstyle +6$ \\
				\cellcolor{blue!100} \color{white} $\scriptstyle Z$
				& \cellcolor{yellow!50} $\scriptstyle +1$
				& \cellcolor{yellow!50} $\scriptstyle -122$
				& \cellcolor{gray!50} $\scriptstyle -183$
				& \cellcolor{yellow!50} $\scriptstyle 0$
				& \cellcolor{yellow!50} $\scriptstyle +60$
				& \cellcolor{yellow!50} $\scriptstyle 0$
				& \cellcolor{yellow!50} $\scriptstyle 0$ 
				& \cellcolor{gray!50} $\scriptstyle -840$  \\
				& & & \includegraphics[width=0.3cm,height=0.3cm]{setacima.jpg} \\
				& & & \scriptsize Entra \\ 
			\end{tabular}
		\end{table}			
	}
	\only<6>
	{
		\begin{table}		
			\begin{tabular}{c c c c c c c c c c c c}
				\cellcolor{blue!100} \color{white} \scriptsize Base 
				&\cellcolor{blue!100} \color{white} \scriptsize Z 
				&\cellcolor{blue!100} \color{white} $\scriptstyle X_1$ 
				&\cellcolor{blue!100} \color{red} $\scriptstyle X_2$ 
				&\cellcolor{blue!100} \color{white}   $\scriptstyle X_3$ 
				&\cellcolor{blue!100} \color{white} $\scriptstyle X_4$ 
				&\cellcolor{blue!100} \color{red}   $\scriptstyle X_5$ 
				&\cellcolor{blue!100} \color{red}   $\scriptstyle X_6$ 
				&\cellcolor{blue!100} \color{white} \scriptsize b
				&
				&
				& \\
				\cellcolor{blue!100} \color{red} $\scriptstyle X_2$
				& \cellcolor{yellow!50} $\scriptstyle 0$
				& \cellcolor{yellow!50} $\scriptstyle -\frac{2}{3}$
				& \cellcolor{yellow!50} $\scriptstyle +1$
				& \cellcolor{yellow!50} $\scriptstyle +\frac{1}{3}$
				& \cellcolor{yellow!50} $\scriptstyle 0$
				& \cellcolor{yellow!50} $\scriptstyle 0$
				& \cellcolor{yellow!50} $\scriptstyle 0$
				& \cellcolor{yellow!50} $\scriptstyle +2$ \\
			    \cellcolor{blue!100} \color{red} $\scriptstyle X_5$
				& \cellcolor{yellow!50} $\scriptstyle 0$
				& \cellcolor{yellow!50} $\scriptstyle +\frac{7}{3}$
				& \cellcolor{yellow!50} $\scriptstyle 0$
				& \cellcolor{yellow!50} $\scriptstyle -\frac{2}{3}$			
				& \cellcolor{yellow!50} $\scriptstyle -1$
				& \cellcolor{yellow!50} $\scriptstyle +1$
				& \cellcolor{yellow!50} $\scriptstyle 0$ 
				& \cellcolor{yellow!50} $\scriptstyle +4$ \\
				\cellcolor{blue!100} \color{red} $\scriptstyle X_6$
				& \cellcolor{yellow!50} $\scriptstyle 0$
				& \cellcolor{yellow!50} $\scriptstyle +\frac{5}{3}$
				& \cellcolor{yellow!50} $\scriptstyle 0$
				& \cellcolor{yellow!50} $\scriptstyle -\frac{1}{3}$
				& \cellcolor{yellow!50} $\scriptstyle 0$
				& \cellcolor{yellow!50} $\scriptstyle 0$
				& \cellcolor{yellow!50} $\scriptstyle +1$
				& \cellcolor{yellow!50} $\scriptstyle +4$ \\
				\cellcolor{blue!100} \color{white} $\scriptstyle Z$
				& \cellcolor{yellow!50} $\scriptstyle +1$
				& \cellcolor{yellow!50} $\scriptstyle -244$
				& \cellcolor{yellow!50} $\scriptstyle 0$
				& \cellcolor{yellow!50} $\scriptstyle +61$
				& \cellcolor{yellow!50} $\scriptstyle +60$
				& \cellcolor{yellow!50} $\scriptstyle 0$
				& \cellcolor{yellow!50} $\scriptstyle 0$ 
				& \cellcolor{yellow!50} $\scriptstyle -474$  \\
			\end{tabular}
		\end{table}			
	}
	\only<7>
	{
		\begin{table}		
			\begin{tabular}{c c c c c c c c c c c c}
				\cellcolor{blue!100} \color{white} \scriptsize Base 
				&\cellcolor{blue!100} \color{white} \scriptsize Z 
				&\cellcolor{blue!100} \color{white} $\scriptstyle X_1$ 
				&\cellcolor{blue!100} \color{red} $\scriptstyle X_2$ 
				&\cellcolor{blue!100} \color{white}   $\scriptstyle X_3$ 
				&\cellcolor{blue!100} \color{white} $\scriptstyle X_4$ 
				&\cellcolor{blue!100} \color{red}   $\scriptstyle X_5$ 
				&\cellcolor{blue!100} \color{red}   $\scriptstyle X_6$ 
				&\cellcolor{blue!100} \color{white} \scriptsize b
				&
				&
				& \\
				\cellcolor{blue!100} \color{red} $\scriptstyle X_2$
				& \cellcolor{yellow!50} $\scriptstyle 0$
				& \cellcolor{gray!50} $\scriptstyle -\frac{2}{3}$
				& \cellcolor{yellow!50} $\scriptstyle +1$
				& \cellcolor{yellow!50} $\scriptstyle +\frac{1}{3}$
				& \cellcolor{yellow!50} $\scriptstyle 0$
				& \cellcolor{yellow!50} $\scriptstyle 0$
				& \cellcolor{yellow!50} $\scriptstyle 0$
				& \cellcolor{yellow!50} $\scriptstyle +2$ \\
			    \cellcolor{blue!100} \color{red} $\scriptstyle X_5$
				& \cellcolor{yellow!50} $\scriptstyle 0$
				& \cellcolor{gray!50} $\scriptstyle +\frac{7}{3}$
				& \cellcolor{yellow!50} $\scriptstyle 0$
				& \cellcolor{yellow!50} $\scriptstyle -\frac{2}{3}$			
				& \cellcolor{yellow!50} $\scriptstyle -1$
				& \cellcolor{yellow!50} $\scriptstyle +1$
				& \cellcolor{yellow!50} $\scriptstyle 0$ 
				& \cellcolor{yellow!50} $\scriptstyle +4$ \\
				\cellcolor{blue!100} \color{red} $\scriptstyle X_6$
				& \cellcolor{yellow!50} $\scriptstyle 0$
				& \cellcolor{gray!50} $\scriptstyle +\frac{5}{3}$
				& \cellcolor{yellow!50} $\scriptstyle 0$
				& \cellcolor{yellow!50} $\scriptstyle -\frac{1}{3}$
				& \cellcolor{yellow!50} $\scriptstyle 0$
				& \cellcolor{yellow!50} $\scriptstyle 0$
				& \cellcolor{yellow!50} $\scriptstyle +1$
				& \cellcolor{yellow!50} $\scriptstyle +4$ \\
				\cellcolor{blue!100} \color{white} $\scriptstyle Z$
				& \cellcolor{yellow!50} $\scriptstyle +1$
				& \cellcolor{gray!50} $\scriptstyle -244$
				& \cellcolor{yellow!50} $\scriptstyle 0$
				& \cellcolor{yellow!50} $\scriptstyle +61$
				& \cellcolor{yellow!50} $\scriptstyle +60$
				& \cellcolor{yellow!50} $\scriptstyle 0$
				& \cellcolor{yellow!50} $\scriptstyle 0$ 
				& \cellcolor{yellow!50} $\scriptstyle -474$  \\
				& & \includegraphics[width=0.3cm,height=0.3cm]{setacima.jpg} \\
				& & \scriptsize Entra \\
			\end{tabular}
		\end{table}			
	}
	\only<8>
	{
		\begin{table}		
			\begin{tabular}{c c c c c c c c c c c c}
				\cellcolor{blue!100} \color{white} \scriptsize Base 
				&\cellcolor{blue!100} \color{white} \scriptsize Z 
				&\cellcolor{blue!100} \color{white} $\scriptstyle X_1$ 
				&\cellcolor{blue!100} \color{red} $\scriptstyle X_2$ 
				&\cellcolor{blue!100} \color{white}   $\scriptstyle X_3$ 
				&\cellcolor{blue!100} \color{white} $\scriptstyle X_4$ 
				&\cellcolor{blue!100} \color{red}   $\scriptstyle X_5$ 
				&\cellcolor{blue!100} \color{red}   $\scriptstyle X_6$ 
				&\cellcolor{blue!100} \color{white} \scriptsize b
				&
				&
				& \\
				\cellcolor{blue!100} \color{red} $\scriptstyle X_2$
				& \cellcolor{yellow!50} $\scriptstyle 0$
				& \cellcolor{gray!50} $\scriptstyle -\frac{2}{3}$
				& \cellcolor{yellow!50} $\scriptstyle +1$
				& \cellcolor{yellow!50} $\scriptstyle +\frac{1}{3}$
				& \cellcolor{yellow!50} $\scriptstyle 0$
				& \cellcolor{yellow!50} $\scriptstyle 0$
				& \cellcolor{yellow!50} $\scriptstyle 0$
				& \cellcolor{gray!50} $\scriptstyle +2$
				& $ \scriptstyle -\frac{3}{4}$ \\
			    \cellcolor{blue!100} \color{red} $\scriptstyle X_5$
				& \cellcolor{yellow!50} $\scriptstyle 0$
				& \cellcolor{gray!50} $\scriptstyle +\frac{7}{3}$
				& \cellcolor{yellow!50} $\scriptstyle 0$
				& \cellcolor{yellow!50} $\scriptstyle -\frac{2}{3}$			
				& \cellcolor{yellow!50} $\scriptstyle -1$
				& \cellcolor{yellow!50} $\scriptstyle +1$
				& \cellcolor{yellow!50} $\scriptstyle 0$ 
				& \cellcolor{gray!50} $\scriptstyle +4$ 
				& $ \scriptstyle \frac{12}{7}$ & \includegraphics[width=0.3cm,height=0.3cm]{setaesquerda.jpg} \\
				\cellcolor{blue!100} \color{red} $\scriptstyle X_6$
				& \cellcolor{yellow!50} $\scriptstyle 0$
				& \cellcolor{gray!50} $\scriptstyle +\frac{5}{3}$
				& \cellcolor{yellow!50} $\scriptstyle 0$
				& \cellcolor{yellow!50} $\scriptstyle -\frac{1}{3}$
				& \cellcolor{yellow!50} $\scriptstyle 0$
				& \cellcolor{yellow!50} $\scriptstyle 0$
				& \cellcolor{yellow!50} $\scriptstyle +1$
				& \cellcolor{gray!50} $\scriptstyle +4$ 
				& $ \scriptstyle \frac{12}{5}$ & \includegraphics[width=0.3cm,height=0.3cm]{setaesquerda.jpg}\\
				\cellcolor{blue!100} \color{white} $\scriptstyle Z$
				& \cellcolor{yellow!50} $\scriptstyle +1$
				& \cellcolor{gray!50} $\scriptstyle -244$
				& \cellcolor{yellow!50} $\scriptstyle 0$
				& \cellcolor{yellow!50} $\scriptstyle +61$
				& \cellcolor{yellow!50} $\scriptstyle +60$
				& \cellcolor{yellow!50} $\scriptstyle 0$
				& \cellcolor{yellow!50} $\scriptstyle 0$ 
				& \cellcolor{gray!50} $\scriptstyle -474$  \\
				& & \includegraphics[width=0.3cm,height=0.3cm]{setacima.jpg} \\
				& & \scriptsize Entra \\
			\end{tabular}
		\end{table}			
	}
	\only<9>
	{
		\begin{table}		
			\begin{tabular}{c c c c c c c c c c c c}
				\cellcolor{blue!100} \color{white} \scriptsize Base 
				&\cellcolor{blue!100} \color{white} \scriptsize Z 
				&\cellcolor{blue!100} \color{white} $\scriptstyle X_1$ 
				&\cellcolor{blue!100} \color{red} $\scriptstyle X_2$ 
				&\cellcolor{blue!100} \color{white}   $\scriptstyle X_3$ 
				&\cellcolor{blue!100} \color{white} $\scriptstyle X_4$ 
				&\cellcolor{blue!100} \color{red}   $\scriptstyle X_5$ 
				&\cellcolor{blue!100} \color{red}   $\scriptstyle X_6$ 
				&\cellcolor{blue!100} \color{white} \scriptsize b
				&
				&
				& \\
				\cellcolor{blue!100} \color{red} $\scriptstyle X_2$
				& \cellcolor{yellow!50} $\scriptstyle 0$
				& \cellcolor{gray!50} $\scriptstyle -\frac{2}{3}$
				& \cellcolor{yellow!50} $\scriptstyle +1$
				& \cellcolor{yellow!50} $\scriptstyle +\frac{1}{3}$
				& \cellcolor{yellow!50} $\scriptstyle 0$
				& \cellcolor{yellow!50} $\scriptstyle 0$
				& \cellcolor{yellow!50} $\scriptstyle 0$
				& \cellcolor{gray!50} $\scriptstyle +2$ \\
			    \cellcolor{blue!100} \color{red} $\scriptstyle X_5$
				& \cellcolor{gray!50} $\scriptstyle 0$
				& \cellcolor{red!50} $\scriptstyle +\frac{7}{3}$
				& \cellcolor{gray!50} $\scriptstyle 0$
				& \cellcolor{gray!50} $\scriptstyle -\frac{2}{3}$			
				& \cellcolor{gray!50} $\scriptstyle -1$
				& \cellcolor{gray!50} $\scriptstyle +1$
				& \cellcolor{gray!50} $\scriptstyle 0$ 
				& \cellcolor{gray!50} $\scriptstyle +4$ 
				& $ \scriptstyle \frac{12}{7}$ & \includegraphics[width=0.3cm,height=0.3cm]{setaesquerda.jpg} \scriptsize Sai \\
				\cellcolor{blue!100} \color{red} $\scriptstyle X_6$
				& \cellcolor{yellow!50} $\scriptstyle 0$
				& \cellcolor{gray!50} $\scriptstyle +\frac{5}{3}$
				& \cellcolor{yellow!50} $\scriptstyle 0$
				& \cellcolor{yellow!50} $\scriptstyle -\frac{1}{3}$
				& \cellcolor{yellow!50} $\scriptstyle 0$
				& \cellcolor{yellow!50} $\scriptstyle 0$
				& \cellcolor{yellow!50} $\scriptstyle +1$
				& \cellcolor{gray!50} $\scriptstyle +4$\\
				\cellcolor{blue!100} \color{white} $\scriptstyle Z$
				& \cellcolor{yellow!50} $\scriptstyle +1$
				& \cellcolor{gray!50} $\scriptstyle -244$
				& \cellcolor{yellow!50} $\scriptstyle 0$
				& \cellcolor{yellow!50} $\scriptstyle +61$
				& \cellcolor{yellow!50} $\scriptstyle +60$
				& \cellcolor{yellow!50} $\scriptstyle 0$
				& \cellcolor{yellow!50} $\scriptstyle 0$ 
				& \cellcolor{gray!50} $\scriptstyle -474$  \\
				& & \includegraphics[width=0.3cm,height=0.3cm]{setacima.jpg} \\
				& & \scriptsize Entra \\
			\end{tabular}
		\end{table}			
	}
	\only<10>
	{
		\begin{table}		
			\begin{tabular}{c c c c c c c c c c c c}
				\cellcolor{blue!100} \color{white} \scriptsize Base 
				&\cellcolor{blue!100} \color{red} \scriptsize Z 
				&\cellcolor{blue!100} \color{red} $\scriptstyle X_1$ 
				&\cellcolor{blue!100} \color{red} $\scriptstyle X_2$ 
				&\cellcolor{blue!100} \color{white}   $\scriptstyle X_3$ 
				&\cellcolor{blue!100} \color{white} $\scriptstyle X_4$ 
				&\cellcolor{blue!100} \color{white}   $\scriptstyle X_5$ 
				&\cellcolor{blue!100} \color{red}   $\scriptstyle X_6$ 
				&\cellcolor{blue!100} \color{white} \scriptsize b \\
				\cellcolor{blue!100} \color{red} $\scriptstyle X_2$
				& \cellcolor{yellow!50} $\scriptstyle 0$
				& \cellcolor{yellow!50} $\scriptstyle 0$
				& \cellcolor{yellow!50} $\scriptstyle +1$
				& \cellcolor{yellow!50} $\scriptstyle +\frac{1}{7}$
				& \cellcolor{yellow!50} $\scriptstyle -\frac{2}{7}$
				& \cellcolor{yellow!50} $\scriptstyle +\frac{2}{7}$
				& \cellcolor{yellow!50} $\scriptstyle 0$
				& \cellcolor{yellow!50} $\scriptstyle +\frac{22}{7}$ \\
			    \cellcolor{blue!100} \color{red} $\scriptstyle X_1$
				& \cellcolor{yellow!50} $\scriptstyle 0$
				& \cellcolor{yellow!50} $\scriptstyle +1$
				& \cellcolor{yellow!50} $\scriptstyle 0$
				& \cellcolor{yellow!50} $\scriptstyle -\frac{2}{7}$			
				& \cellcolor{yellow!50} $\scriptstyle -\frac{3}{7}$
				& \cellcolor{yellow!50} $\scriptstyle +\frac{3}{7}$
				& \cellcolor{yellow!50} $\scriptstyle 0$ 
				& \cellcolor{yellow!50} $\scriptstyle +\frac{12}{7}$  \\
				\cellcolor{blue!100} \color{red} $\scriptstyle X_6$
				& \cellcolor{yellow!50} $\scriptstyle 0$
				& \cellcolor{yellow!50} $\scriptstyle 0$
				& \cellcolor{yellow!50} $\scriptstyle 0$
				& \cellcolor{yellow!50} $\scriptstyle +\frac{1}{7}$
				& \cellcolor{yellow!50} $\scriptstyle +\frac{5}{7}$
				& \cellcolor{yellow!50} $\scriptstyle -\frac{5}{7}$
				& \cellcolor{yellow!50} $\scriptstyle +1$
				& \cellcolor{yellow!50} $\scriptstyle +\frac{8}{7}$ \\
				\cellcolor{blue!100} \color{white} $\scriptstyle Z$
				& \cellcolor{yellow!50} $\scriptstyle +1$
				& \cellcolor{yellow!50} $\scriptstyle 0$
				& \cellcolor{yellow!50} $\scriptstyle 0$
				& \cellcolor{yellow!50} $\scriptstyle -\frac{61}{7}$
				& \cellcolor{yellow!50} $\scriptstyle -\frac{312}{7}$
				& \cellcolor{yellow!50} $\scriptstyle +\frac{732}{7}$
				& \cellcolor{yellow!50} $\scriptstyle 0$ 
				& \cellcolor{yellow!50} $\scriptstyle -\frac{390}{7}$  \\
			\end{tabular}
		\end{table}			
	}
	\only<11>
	{
		\begin{table}		
			\begin{tabular}{c c c c c c c c c c c c}
				\cellcolor{blue!100} \color{white} \scriptsize Base 
				&\cellcolor{blue!100} \color{red} \scriptsize Z 
				&\cellcolor{blue!100} \color{red} $\scriptstyle X_1$ 
				&\cellcolor{blue!100} \color{red} $\scriptstyle X_2$ 
				&\cellcolor{blue!100} \color{white}   $\scriptstyle X_3$ 
				&\cellcolor{blue!100} \color{white} $\scriptstyle X_4$ 
				&\cellcolor{blue!100} \color{white}   $\scriptstyle X_5$ 
				&\cellcolor{blue!100} \color{red}   $\scriptstyle X_6$ 
				&\cellcolor{blue!100} \color{white} \scriptsize b \\
				\cellcolor{blue!100} \color{red} $\scriptstyle X_2$
				& \cellcolor{yellow!50} $\scriptstyle 0$
				& \cellcolor{yellow!50} $\scriptstyle 0$
				& \cellcolor{yellow!50} $\scriptstyle +1$
				& \cellcolor{yellow!50} $\scriptstyle +\frac{1}{7}$
				& \cellcolor{gray!50} $\scriptstyle -\frac{2}{7}$
				& \cellcolor{yellow!50} $\scriptstyle +\frac{2}{7}$
				& \cellcolor{yellow!50} $\scriptstyle 0$
				& \cellcolor{yellow!50} $\scriptstyle +\frac{22}{7}$ \\
			    \cellcolor{blue!100} \color{red} $\scriptstyle X_1$
				& \cellcolor{yellow!50} $\scriptstyle 0$
				& \cellcolor{yellow!50} $\scriptstyle +1$
				& \cellcolor{yellow!50} $\scriptstyle 0$
				& \cellcolor{yellow!50} $\scriptstyle -\frac{2}{7}$			
				& \cellcolor{gray!50} $\scriptstyle -\frac{3}{7}$
				& \cellcolor{yellow!50} $\scriptstyle +\frac{3}{7}$
				& \cellcolor{yellow!50} $\scriptstyle 0$ 
				& \cellcolor{yellow!50} $\scriptstyle +\frac{12}{7}$  \\
				\cellcolor{blue!100} \color{red} $\scriptstyle X_6$
				& \cellcolor{yellow!50} $\scriptstyle 0$
				& \cellcolor{yellow!50} $\scriptstyle 0$
				& \cellcolor{yellow!50} $\scriptstyle 0$
				& \cellcolor{yellow!50} $\scriptstyle +\frac{1}{7}$
				& \cellcolor{gray!50} $\scriptstyle +\frac{5}{7}$
				& \cellcolor{yellow!50} $\scriptstyle -\frac{5}{7}$
				& \cellcolor{yellow!50} $\scriptstyle +1$
				& \cellcolor{yellow!50} $\scriptstyle +\frac{8}{7}$ \\
				\cellcolor{blue!100} \color{white} $\scriptstyle Z$
				& \cellcolor{yellow!50} $\scriptstyle +1$
				& \cellcolor{yellow!50} $\scriptstyle 0$
				& \cellcolor{yellow!50} $\scriptstyle 0$
				& \cellcolor{yellow!50} $\scriptstyle -\frac{61}{7}$
				& \cellcolor{gray!50} $\scriptstyle -\frac{312}{7}$
				& \cellcolor{yellow!50} $\scriptstyle +\frac{732}{7}$
				& \cellcolor{yellow!50} $\scriptstyle 0$ 
				& \cellcolor{yellow!50} $\scriptstyle -\frac{390}{7}$  \\
				& & & & & \includegraphics[width=0.3cm,height=0.3cm]{setacima.jpg} \\
				& & & & & \scriptsize Entra \\
			\end{tabular}
		\end{table}			
	}
	\only<12>
	{
		\begin{table}		
			\begin{tabular}{c c c c c c c c c c c c}
				\cellcolor{blue!100} \color{white} \scriptsize Base 
				&\cellcolor{blue!100} \color{red} \scriptsize Z 
				&\cellcolor{blue!100} \color{red} $\scriptstyle X_1$ 
				&\cellcolor{blue!100} \color{red} $\scriptstyle X_2$ 
				&\cellcolor{blue!100} \color{white}   $\scriptstyle X_3$ 
				&\cellcolor{blue!100} \color{white} $\scriptstyle X_4$ 
				&\cellcolor{blue!100} \color{white}   $\scriptstyle X_5$ 
				&\cellcolor{blue!100} \color{red}   $\scriptstyle X_6$ 
				&\cellcolor{blue!100} \color{white} \scriptsize b \\
				\cellcolor{blue!100} \color{red} $\scriptstyle X_2$
				& \cellcolor{yellow!50} $\scriptstyle 0$
				& \cellcolor{yellow!50} $\scriptstyle 0$
				& \cellcolor{yellow!50} $\scriptstyle +1$
				& \cellcolor{yellow!50} $\scriptstyle +\frac{1}{7}$
				& \cellcolor{gray!50} $\scriptstyle -\frac{2}{7}$
				& \cellcolor{yellow!50} $\scriptstyle +\frac{2}{7}$
				& \cellcolor{yellow!50} $\scriptstyle 0$
				& \cellcolor{gray!50} $\scriptstyle +\frac{22}{7}$
				& $ \scriptstyle -11$ \\
			    \cellcolor{blue!100} \color{red} $\scriptstyle X_1$
				& \cellcolor{yellow!50} $\scriptstyle 0$
				& \cellcolor{yellow!50} $\scriptstyle +1$
				& \cellcolor{yellow!50} $\scriptstyle 0$
				& \cellcolor{yellow!50} $\scriptstyle -\frac{2}{7}$			
				& \cellcolor{gray!50} $\scriptstyle -\frac{3}{7}$
				& \cellcolor{yellow!50} $\scriptstyle +\frac{3}{7}$
				& \cellcolor{yellow!50} $\scriptstyle 0$ 
				& \cellcolor{gray!50} $\scriptstyle +\frac{12}{7}$
				& $ \scriptstyle -4 $  \\
				\cellcolor{blue!100} \color{red} $\scriptstyle X_6$
				& \cellcolor{yellow!50} $\scriptstyle 0$
				& \cellcolor{yellow!50} $\scriptstyle 0$
				& \cellcolor{yellow!50} $\scriptstyle 0$
				& \cellcolor{yellow!50} $\scriptstyle +\frac{1}{7}$
				& \cellcolor{gray!50} $\scriptstyle +\frac{5}{7}$
				& \cellcolor{yellow!50} $\scriptstyle -\frac{5}{7}$
				& \cellcolor{yellow!50} $\scriptstyle +1$
				& \cellcolor{gray!50} $\scriptstyle +\frac{8}{7}$
				& $ \scriptstyle +\frac{8}{5}$ \includegraphics[width=0.3cm,height=0.3cm]{setaesquerda.jpg}\\
				\cellcolor{blue!100} \color{white} $\scriptstyle Z$
				& \cellcolor{yellow!50} $\scriptstyle +1$
				& \cellcolor{yellow!50} $\scriptstyle 0$
				& \cellcolor{yellow!50} $\scriptstyle 0$
				& \cellcolor{yellow!50} $\scriptstyle -\frac{61}{7}$
				& \cellcolor{gray!50} $\scriptstyle -\frac{312}{7}$
				& \cellcolor{yellow!50} $\scriptstyle +\frac{732}{7}$
				& \cellcolor{yellow!50} $\scriptstyle 0$ 
				& \cellcolor{gray!50} $\scriptstyle -\frac{390}{7}$  \\
				& & & & & \includegraphics[width=0.3cm,height=0.3cm]{setacima.jpg} \\
				& & & & & \scriptsize Entra \\
			\end{tabular}
		\end{table}			
	}
	\only<13>
	{
		\begin{table}		
			\begin{tabular}{c c c c c c c c c c c c}
				\cline{1-8} 
				\cellcolor{blue!100} \color{white} \scriptsize Base 
				&\cellcolor{blue!100} \color{red} \scriptsize Z 
				&\cellcolor{blue!100} \color{red} $\scriptstyle X_1$ 
				&\cellcolor{blue!100} \color{red} $\scriptstyle X_2$ 
				&\cellcolor{blue!100} \color{white}   $\scriptstyle X_3$ 
				&\cellcolor{blue!100} \color{white} $\scriptstyle X_4$ 
				&\cellcolor{blue!100} \color{white}   $\scriptstyle X_5$ 
				&\cellcolor{blue!100} \color{red}   $\scriptstyle X_6$ 
				&\cellcolor{blue!100} \color{white} \scriptsize b \\
				\cellcolor{blue!100} \color{red} $\scriptstyle X_2$
				& \cellcolor{yellow!50} $\scriptstyle 0$
				& \cellcolor{yellow!50} $\scriptstyle 0$
				& \cellcolor{yellow!50} $\scriptstyle +1$
				& \cellcolor{yellow!50} $\scriptstyle +\frac{1}{7}$
				& \cellcolor{gray!50} $\scriptstyle -\frac{2}{7}$
				& \cellcolor{yellow!50} $\scriptstyle +\frac{2}{7}$
				& \cellcolor{yellow!50} $\scriptstyle 0$
				& \cellcolor{gray!50} $\scriptstyle +\frac{22}{7}$
				& $ \scriptstyle -11$ \\
			    \cellcolor{blue!100} \color{red} $\scriptstyle X_1$
				& \cellcolor{yellow!50} $\scriptstyle 0$
				& \cellcolor{yellow!50} $\scriptstyle +1$
				& \cellcolor{yellow!50} $\scriptstyle 0$
				& \cellcolor{yellow!50} $\scriptstyle -\frac{2}{7}$			
				& \cellcolor{gray!50} $\scriptstyle -\frac{3}{7}$
				& \cellcolor{yellow!50} $\scriptstyle +\frac{3}{7}$
				& \cellcolor{yellow!50} $\scriptstyle 0$ 
				& \cellcolor{gray!50} $\scriptstyle +\frac{12}{7}$
				& $ \scriptstyle -4 $  \\
				\cellcolor{blue!100} \color{red} $\scriptstyle X_6$
				& \cellcolor{gray!50} $\scriptstyle 0$
				& \cellcolor{gray!50} $\scriptstyle 0$
				& \cellcolor{gray!50} $\scriptstyle 0$
				& \cellcolor{gray!50} $\scriptstyle +\frac{1}{7}$
				& \cellcolor{red!50} $\scriptstyle +\frac{5}{7}$
				& \cellcolor{gray!50} $\scriptstyle -\frac{5}{7}$
				& \cellcolor{gray!50} $\scriptstyle +1$
				& \cellcolor{gray!50} $\scriptstyle +\frac{8}{7}$
				& $ \scriptstyle +\frac{8}{5}$ \includegraphics[width=0.3cm,height=0.3cm]{setaesquerda.jpg} \scriptsize Sai\\
				\cellcolor{blue!100} \color{white} $\scriptstyle Z$
				& \cellcolor{yellow!50} $\scriptstyle +1$
				& \cellcolor{yellow!50} $\scriptstyle 0$
				& \cellcolor{yellow!50} $\scriptstyle 0$
				& \cellcolor{yellow!50} $\scriptstyle -\frac{61}{7}$
				& \cellcolor{gray!50} $\scriptstyle -\frac{312}{7}$
				& \cellcolor{yellow!50} $\scriptstyle +\frac{732}{7}$
				& \cellcolor{yellow!50} $\scriptstyle 0$ 
				& \cellcolor{gray!50} $\scriptstyle -\frac{390}{7}$  \\
				& & & & & \includegraphics[width=0.3cm,height=0.3cm]{setacima.jpg} \\
				& & & & & \scriptsize Entra \\
			\end{tabular}
		\end{table}			
	}
	\only<14>
	{
		\begin{table}		
			\begin{tabular}{c c c c c c c c c c c c}
				\cellcolor{blue!100} \color{white} \scriptsize Base 
				&\cellcolor{blue!100} \color{red} \scriptsize Z 
				&\cellcolor{blue!100} \color{red}   $\scriptstyle X_1$ 
				&\cellcolor{blue!100} \color{red}   $\scriptstyle X_2$ 
				&\cellcolor{blue!100} \color{white} $\scriptstyle X_3$ 
				&\cellcolor{blue!100} \color{red}   $\scriptstyle X_4$ 
				&\cellcolor{blue!100} \color{white} $\scriptstyle X_5$ 
				&\cellcolor{blue!100} \color{white} $\scriptstyle X_6$ 
				&\cellcolor{blue!100} \color{white} \scriptsize b \\
				\cellcolor{blue!100} \color{red} $\scriptstyle X_2$
				& \cellcolor{yellow!50} $\scriptstyle 0$
				& \cellcolor{yellow!50} $\scriptstyle 0$
				& \cellcolor{yellow!50} $\scriptstyle +1$
				& \cellcolor{yellow!50} $\scriptstyle +\frac{1}{5}$
				& \cellcolor{yellow!50} $\scriptstyle 0$
				& \cellcolor{yellow!50} $\scriptstyle 0$
				& \cellcolor{yellow!50} $\scriptstyle +\frac{2}{5}$
				& \cellcolor{yellow!50} $\scriptstyle +\frac{18}{5}$ \\
			    \cellcolor{blue!100} \color{red} $\scriptstyle X_1$
				& \cellcolor{yellow!50} $\scriptstyle 0$
				& \cellcolor{yellow!50} $\scriptstyle +1$
				& \cellcolor{yellow!50} $\scriptstyle 0$
				& \cellcolor{yellow!50} $\scriptstyle -\frac{1}{5}$			
				& \cellcolor{yellow!50} $\scriptstyle 0$
				& \cellcolor{yellow!50} $\scriptstyle 0$
				& \cellcolor{yellow!50} $\scriptstyle +\frac{3}{5}$ 
				& \cellcolor{yellow!50} $\scriptstyle +\frac{12}{5}$  \\
				\cellcolor{blue!100} \color{red} $\scriptstyle X_4$
				& \cellcolor{yellow!50} $\scriptstyle 0$
				& \cellcolor{yellow!50} $\scriptstyle 0$
				& \cellcolor{yellow!50} $\scriptstyle 0$
				& \cellcolor{yellow!50} $\scriptstyle +\frac{1}{5}$
				& \cellcolor{yellow!50} $\scriptstyle +1$
				& \cellcolor{yellow!50} $\scriptstyle -1$
				& \cellcolor{yellow!50} $\scriptstyle +\frac{7}{5}$
				& \cellcolor{yellow!50} $\scriptstyle +\frac{8}{5}$ \\
				\cellcolor{blue!100} \color{white} $\scriptstyle Z$
				& \cellcolor{yellow!50} $\scriptstyle +1$
				& \cellcolor{yellow!50} $\scriptstyle 0$
				& \cellcolor{yellow!50} $\scriptstyle 0$
				& \cellcolor{yellow!50} $\scriptstyle +\frac{1}{5}$
				& \cellcolor{yellow!50} $\scriptstyle 0$
				& \cellcolor{yellow!50} $\scriptstyle +60$
				& \cellcolor{yellow!50} $\scriptstyle +\frac{312}{5}$ 
				& \cellcolor{yellow!50} $\scriptstyle +\frac{78}{5}$  \\
			\end{tabular}
		\end{table}			
	}
	
	

	% % % % % % % % % % % % % % % % % % % % % % % % % % % % % % % % % % % % % % % % % % % % % % %
	\only<1>
	{
		\begin{columns}
			\begin{column}{5cm}
				\begin{itemize}
				\item Maior coeficiente função objetivo orginal: 3
				\item $ M = 20 \times 3 = 60$
				\end{itemize}
			\end{column}
			\begin{column}{5cm}

			\end{column}
		\end{columns}
	}	
	\only<2>
	{
		\begin{columns}
			\begin{column}{5cm}
				\begin{mdframed}[backgroundcolor=olive!80]
					Verificar se a solução é ótima! (Linha de Z).
				\end{mdframed}
			\end{column}
			\begin{column}{5cm}

			\end{column}
		\end{columns}
	}	
	\only<3>
	{
		\begin{columns}
			\begin{column}{5cm}
				\begin{mdframed}[backgroundcolor=orange!80]
					Escolhe variável com coeficiente mais negativo na FOB para entrar na base.
				\end{mdframed}
			\end{column}
			\begin{column}{5cm}

			\end{column}
		\end{columns}
	}	
	\only<4>
	{
		\begin{columns}
			\begin{column}{5cm}
				\begin{mdframed}[backgroundcolor=olive!80]
					Para cada linha do Tableau calcular razão $\frac{b_i}{\text{coluna}}$ e verificar se há razão positiva e finita.
				\end{mdframed}
			\end{column}
			\begin{column}{5cm}

			\end{column}
		\end{columns}
	}	

	\only<5>
	{
		\begin{columns}
			\begin{column}{5cm}
				\begin{mdframed}[backgroundcolor=orange!80]
					A linha com a menor razão positiva finita sairá da base.
				\end{mdframed}
			\end{column}
			\begin{column}{5cm}
				$
					\begin{matrix}
						\scriptstyle \text{LINHA}'_1 = \frac{1}{3} \times \text{LINHA}_1 \\
						\scriptstyle \text{LINHA}'_2 = \text{LINHA}_2 - \frac{2}{3} \times \text{LINHA}_1 \\
						\scriptstyle \text{LINHA}'_3 = \text{LINHA}_3 - \frac{1}{3} \times \text{LINHA}_1 \\
						\scriptstyle \text{LINHA}'_4 = \text{LINHA}_4 + 61 \times \text{LINHA}_1 \\
					\end{matrix}
				$
			\end{column}
		\end{columns}
	}	


	\only<6>
	{
		\begin{columns}
			\begin{column}{5cm}
				\begin{mdframed}[backgroundcolor=olive!80]
					Verificar se a solução é ótima! (Linha de Z).
				\end{mdframed}
			\end{column}
			\begin{column}{5cm}

			\end{column}
		\end{columns}
	}	
	\only<7>
	{
		\begin{columns}
			\begin{column}{5cm}
				\begin{mdframed}[backgroundcolor=orange!80]
					Escolhe variável com coeficiente mais negativo na FOB para entrar na base.
				\end{mdframed}
			\end{column}
			\begin{column}{5cm}

			\end{column}
		\end{columns}
	}	
	\only<8>
	{
		\begin{columns}
			\begin{column}{5cm}
				\begin{mdframed}[backgroundcolor=olive!80]
					Para cada linha do Tableau calcular razão $\frac{b_i}{\text{coluna}}$ e verificar se há razão positiva e finita.
				\end{mdframed}
			\end{column}
			\begin{column}{5cm}

			\end{column}
		\end{columns}
	}	

	\only<9>
	{
		\begin{columns}
			\begin{column}{5cm}
				\begin{mdframed}[backgroundcolor=orange!80]
					A linha com a menor razão positiva finita sairá da base.
				\end{mdframed}
			\end{column}
			\begin{column}{5cm}
				$
					\begin{matrix}
						\scriptstyle \text{LINHA}'_1 = \text{LINHA}_1 + \frac{2}{7} \times \text{LINHA}_2\\
						\scriptstyle \text{LINHA}'_2 = \frac{3}{7} \times \text{LINHA}_2 \\
						\scriptstyle \text{LINHA}'_3 = \text{LINHA}_3 - \frac{5}{7} \times \text{LINHA}_2 \\
						\scriptstyle \text{LINHA}'_4 = \text{LINHA}_4 + \frac{732}{7} \times \text{LINHA}_2 \\
					\end{matrix}
				$
			\end{column}
		\end{columns}
	}	

	\only<10>
	{
		\begin{columns}
			\begin{column}{5cm}
				\begin{mdframed}[backgroundcolor=olive!80]
					Verificar se a solução é ótima! (Linha de Z).
				\end{mdframed}
			\end{column}
			\begin{column}{5cm}

			\end{column}
		\end{columns}
	}	
	\only<11>
	{
		\begin{columns}
			\begin{column}{5cm}
				\begin{mdframed}[backgroundcolor=orange!80]
					Escolhe variável com coeficiente mais negativo na FOB para entrar na base.
				\end{mdframed}
			\end{column}
			\begin{column}{5cm}

			\end{column}
		\end{columns}
	}	
	\only<12>
	{
		\begin{columns}
			\begin{column}{5cm}
				\begin{mdframed}[backgroundcolor=olive!80]
					Para cada linha do Tableau calcular razão $\frac{b_i}{\text{coluna}}$ e verificar se há razão positiva e finita.
				\end{mdframed}
			\end{column}
			\begin{column}{5cm}

			\end{column}
		\end{columns}
	}	

	\only<13>
	{
		\begin{columns}
			\begin{column}{5cm}
				\begin{mdframed}[backgroundcolor=orange!80]
					A linha com a menor razão positiva finita sairá da base.
				\end{mdframed}
			\end{column}
			\begin{column}{5cm}
				$
					\begin{matrix}
						\scriptstyle \text{LINHA}'_1 = \text{LINHA}_1 + \frac{2}{5} \times \text{LINHA}_3\\
						\scriptstyle \text{LINHA}'_2 = \text{LINHA}_2 + \frac{3}{5} \times \text{LINHA}_3\\
						\scriptstyle \text{LINHA}'_3 = \frac{7}{5} \times \text{LINHA}_3  \\
						\scriptstyle \text{LINHA}'_4 = \text{LINHA}_4 + \frac{312}{5} \times \text{LINHA}_3 \\
					\end{matrix}
				$
			\end{column}
		\end{columns}
	}	


	
	\only<14>
	{
		\begin{columns}
			\begin{column}{5cm}
				\begin{mdframed}[backgroundcolor=red!80]
					\centering
					OPTIMAL SOLUTION FOUND !!!!
				\end{mdframed}
			\end{column}
			\begin{column}{5cm}
				\begin{table}
					\begin{tabular}{ c | c | c |}
						\hline
						\cellcolor{blue!100} \color{white}\scriptsize Tipo     & 
						\cellcolor{blue!100} \color{white} \scriptsize Variável &  
						\cellcolor{blue!100} \color{white} \scriptsize Valor \\
						\cellcolor{green!80}  & 
						\cellcolor{green!80} $\scriptstyle X_1$   &  
						\cellcolor{green!80} $\scriptstyle \frac{12}{5}$   \\
						\cellcolor{green!80} VB & 
						\cellcolor{green!80} $\scriptstyle X_2$   &  
						\cellcolor{green!80} $\scriptstyle \frac{18}{5}$   \\
						\cellcolor{green!80} & 
						\cellcolor{green!80} $\scriptstyle X_4$   &  
						\cellcolor{green!100} $\scriptstyle \frac{8}{5}$   \\
						\cellcolor{yellow!50} & 
						\cellcolor{yellow!50} $ \scriptstyle X_3$   &  
						\cellcolor{yellow!50} $\scriptstyle 0$   \\
						\cellcolor{yellow!50} VNB & 
						\cellcolor{yellow!50} $\scriptstyle X_5$   &  
						\cellcolor{yellow!50} $\scriptstyle 0$   \\
						\cellcolor{yellow!50} & 
						\cellcolor{yellow!50} $\scriptstyle X_6$   &  
						\cellcolor{yellow!50} $\scriptstyle 0$   \\						
						\cellcolor{green!80} FOB 				 & 
						\cellcolor{green!80} $\scriptstyle Z$     &  
						\cellcolor{green!80} $\scriptstyle \frac{78}{5}$  \\
						\hline		
					\end{tabular}
				\end{table}
			\end{column}
		\end{columns}
	}	
\end{frame}

\section{Análise de Sensibilidade ou Análise de Pós-Otimização}
\subsection{Exemplo Tableau-Simplex}

\begin{frame}
	\frametitle{O Problema dos Cereais - Matlab}
	\centering
	\includegraphics[width=12cm,height=7cm]{Prob_Agricultor.png}
\end{frame}
 
\begin{frame}
	\frametitle{O Problema dos Cereais}
	\begin{columns}
		\begin{column}{4cm}
			\scriptsize
			\begin{table}
				\begin{tabular}{c}
					\cellcolor{red!50} $ \max z = 600x_1 + 800x_2 $ \\
					\cellcolor{red!50} Sujeito a: \\
					\cellcolor{red!50} $x_1 + x_2 \le 100$ \\
					\cellcolor{red!50} $3x_1 + 2x_2 \le 240$ \\
					\cellcolor{red!50} $x_1 \le 60 $ \\
					\cellcolor{red!50} $x_2 \le 80 $ \\
				    \cellcolor{red!50} $x_1, x_2 \ge 0 $ \\
				\end{tabular}
			\end{table}
		\end{column} \pause
		\begin{column}{3cm}
			\centering
			\textbf{Forma Padrão} \\
			\includegraphics[width=3cm,height=0.5cm]{seta.png} \\
			\textbf{Tableau Simplex} \\
		\end{column}
		\begin{column}{4cm}
			\scriptsize
			\begin{table}
				\begin{tabular}{c}
					\cellcolor{red!50} $ \max z = 600x_1 + 800x_2 $ \\
					\cellcolor{red!50} Sujeito a: \\
					\cellcolor{red!50} $x_1 + x_2 {\color{green}+ S_1 =}  100$ \\
					\cellcolor{red!50} $3x_1 + 2x_2 {\color{green}+ S_2 =}  240$ \\
					\cellcolor{red!50} $x_1 {\color{green}+ S_3 =}  60 $  \\
					\cellcolor{red!50} $x_2 {\color{green}+ S_4 =}  80 $ \\
				    \cellcolor{red!50} $x_1, x_2 \ge 0 $ \\
				\end{tabular}
			\end{table}
		\end{column} \pause
	\end{columns}
	
	\begin{table}
		\caption{{\textbf{Tableau Simplex Final Processo Iterativo}}}
		\begin{tabular}{c c c c c c c c c}
			\cellcolor{blue} {\color{white} Base} &
			\cellcolor{blue} {\color{white} $Z$} &
			\cellcolor{blue} {\color{white} $x_1$} &
			\cellcolor{blue} {\color{white} $x_2$} &
			\cellcolor{blue} {\color{white} $S_1$} &
			\cellcolor{blue} {\color{white} $S_2$} &
			\cellcolor{blue} {\color{white} $S_3$} &
			\cellcolor{blue} {\color{white} $S_4$} &
			\cellcolor{blue} {\color{white} $B$} \\
			\cellcolor{blue} {\color{white} $X_1$} &
			\cellcolor{yellow} $0$& 
			\cellcolor{yellow} $1$& 
			\cellcolor{yellow} $0$& 
			\cellcolor{yellow} $1$& 
			\cellcolor{yellow} $0$& 
			\cellcolor{yellow} $0$& 
			\cellcolor{yellow} $-1$&
			\cellcolor{yellow} $20$\\
			\cellcolor{blue} {\color{white} $S_2$} &
			\cellcolor{yellow} $0$& 
			\cellcolor{yellow} $0$& 
			\cellcolor{yellow} $0$& 
			\cellcolor{yellow} $-3$& 
			\cellcolor{yellow} $1$& 
			\cellcolor{yellow} $0$& 
			\cellcolor{yellow} $1$&
			\cellcolor{yellow} $20$\\
			\cellcolor{blue} {\color{white} $S_3$} &
			\cellcolor{yellow} $0$& 
			\cellcolor{yellow} $0$& 
			\cellcolor{yellow} $0$& 
			\cellcolor{yellow} $-2$& 
			\cellcolor{yellow} $0$& 
			\cellcolor{yellow} $1$& 
			\cellcolor{yellow} $1$&
			\cellcolor{yellow} $40$\\
			\cellcolor{blue} {\color{white} $X_2$} &
			\cellcolor{yellow} $0$& 
			\cellcolor{yellow} $0$& 
			\cellcolor{yellow} $1$& 
			\cellcolor{yellow} $0$& 
			\cellcolor{yellow} $0$& 
			\cellcolor{yellow} $0$& 
			\cellcolor{yellow} $1$&
			\cellcolor{yellow} $80$\\
			\cellcolor{blue} {\color{white} $Z$} &
			\cellcolor{yellow} $1$& 
			\cellcolor{yellow} $0$& 
			\cellcolor{yellow} $0$& 
			\cellcolor{yellow} $600$& 
			\cellcolor{yellow} $0$& 
			\cellcolor{yellow} $0$& 
			\cellcolor{yellow} $200$&
			\cellcolor{yellow} $76000$\\
		\end{tabular}
	\end{table}
\end{frame}

\subsection{Variáveis Duais x Tableau Simplex}
\begin{frame}
	\frametitle{O Problema dos Cereais - Tableau Final - Linha FOB}
	\centering
	\begin{table}
		\scriptsize
		\begin{tabular}{c c c c c c c c c}
			\cellcolor{blue} {\color{white} Base} &
			\cellcolor{blue} {\color{white} $Z$} &
			\cellcolor{blue} {\color{white} $x_1$} &
			\cellcolor{blue} {\color{white} $x_2$} &
			\cellcolor{blue} {\color{red} $\mathbf{S_1}$} &
			\cellcolor{blue} {\color{white} $S_2$} &
			\cellcolor{blue} {\color{white} $S_3$} &
			\cellcolor{blue} {\color{red} $\mathbf{S_4}$} &
			\cellcolor{blue} {\color{white} $B$} \\
			\cellcolor{blue} {\color{white} $Z$} &
			\cellcolor{yellow} $1$& 
			\cellcolor{yellow} $0$& 
			\cellcolor{yellow} $0$& 
			\cellcolor{red} $600$& 
			\cellcolor{yellow} $0$& 
			\cellcolor{yellow} $0$& 
			\cellcolor{red} $200$&
			\cellcolor{yellow} $76000$\\
		\end{tabular}
	\end{table} 
	\includegraphics[width=7.5cm,height=5.5cm]{Prob_Agricultor.png}\\
	\scriptsize
	\underline{\textbf{Conclusão}:} Os coeficientes da VNB da FOB (Tableau Ótimo) são referentes aos valores das variáveis duais e suas respectivas restrições.
\end{frame}

\begin{frame}
	\frametitle{O Problema dos Cereais}
	\centering
	\includegraphics[width=8.5cm,height=7cm]{anima_22.jpg}
\end{frame}


\begin{frame}
	\frametitle{Programação Linear - Análise de Sensibilidade}
	\begin{block}{Justificativas para a Análise Pós-Otimização}
		\begin{itemize}
			\item O problema em estudo exige grande tempo de processamento e já foi resolvido anteriormente.
			\item Modificação ou ajuste de algum parâmetro do modelo.
			\item Estudo do impacto da variação de alguns parâmetros do modelo na função objetivo do problema.
		\end{itemize}
	\end{block}
\end{frame}

\begin{frame}
	\frametitle{Programação Linear - Análise de Sensibilidade}
	\begin{exampleblock}{Análise Pós-Otimização pode ser feita em relação:}
		\begin{itemize}
			\item Variação dos recursos das restrições
			\item Eliminação das variáveis
			\item Eliminação das restrições
			\item Variação dos coeficientes da FOB
		\end{itemize}
	\end{exampleblock}
\end{frame}


\subsection{Eliminação de Variáveis}
\begin{frame}
	\frametitle{Eliminação de Variáveis}
	\begin{block}{Quais variáveis podem ser eliminadas?}
		\begin{itemize}
			\item Variáveis nulas podem ser eliminadas (VNB e/ou VB nulas) sem afetar a solução ótima do problema.
			\item Variáveis básicas (VB) \underline{não nulas} {\color{red}não podem} ser eliminadas. Caso sejam, deve-se resolver o problema novamente.
		\end{itemize}
	\end{block}
\end{frame}

\begin{frame}
	\frametitle{Eliminação de Variáveis}
	\begin{columns}
		\begin{column}{4cm}
			\only<1->
			{
				\centering
				\scriptsize
				\begin{table}
					\begin{tabular}{c}
						\cellcolor{red!50} $ \max z = 600x_1 + 800x_2  + x_3 $\\
						\cellcolor{red!50} Sujeito a: \\
						\cellcolor{red!50} $x_1 + x_2 + x_3 \le 100$ \\
						\cellcolor{red!50} $3x_1 + 2x_2 \le 240$ \\
						\cellcolor{red!50} $x_1 \le 60 $ \\
						\cellcolor{red!50} $x_2 \le 80 $ \\
					    \cellcolor{red!50} $x_1, x_2, x_3 \ge 0 $ \\
					\end{tabular}
				\end{table}
			}
		\end{column}
		\begin{column}{3cm}
			\only<2->
			{		
				\centering
				\includegraphics[width=2cm,height=2cm]{solver_matlab.png} \\
				\includegraphics[width=1.5cm,height=1cm]{seta_azul_direita.jpg} \\
			}
		\end{column}
		\begin{column}{4cm}
			\only<2->
			{		
				\centering
				\includegraphics[width=3cm,height=3cm]{Elimina_Matlab_1.png}
			}
		\end{column}
	\end{columns}
	\begin{columns}
		\begin{column}{4cm}
			\only<4->
			{		
				\centering
				\includegraphics[width=1cm,height=1.5cm]{seta_azul_cima.jpg}		
			}
		\end{column}
		\begin{column}{3cm}
		\end{column}
		\begin{column}{4cm}
			\only<3->
			{		
				\centering
				%\includegraphics[width=1cm,height=1.5cm]{seta_azul_baixo.jpg}
				\includegraphics[width=2cm,height=1cm]{tesoura.png} {\color{red}$X_3$}
			}
		\end{column}
	\end{columns}
	\begin{columns}
		\begin{column}{4cm}
			\only<4->
			{		
				\centering
				\includegraphics[width=2.5cm,height=2.5cm]{Elimina_Matlab_2.png}
			}
		\end{column} 
		\begin{column}{3cm}
			\only<4->
			{		
				\centering
				\includegraphics[width=1.5cm,height=1cm]{seta_azul_esquerda.jpg} \\		
			}
		\end{column}
		\begin{column}{4cm}
			\only<3->
			{		
				\centering
				\scriptsize
				\begin{table}
					\begin{tabular}{c}
						\cellcolor{olive!50} $ \max z = 600x_1 + 800x_2 $\\
						\cellcolor{olive!50} Sujeito a: \\
						\cellcolor{olive!50} $x_1 + x_2 \le 100$ \\
						\cellcolor{olive!50} $3x_1 + 2x_2 \le 240$ \\
						\cellcolor{olive!50} $x_1 \le 60 $ \\
						\cellcolor{olive!50} $x_2 \le 80 $ \\
					    \cellcolor{olive!50} $x_1, x_2 \ge 0 $ \\
					\end{tabular}
				\end{table}
			}
		\end{column} 
	\end{columns}	
\end{frame}

\subsection{Eliminação de Restrições}
\begin{frame}
	\frametitle{Eliminação de Restrições}
	\begin{itemize}
		\item {A restrição a ser eliminada é uma igualdade:}
			\begin{itemize}
			\item[$\diamond$] \textbf{Neste caso, deve-se resolver o problema novamente.}
			\end{itemize} 
		\item {A restrição a ser eliminada é uma desigualdade. Neste caso, deve-se verificar se a restrição é ativa ou não}
			\begin{itemize}
			\item[$\diamond$] \textbf{\underline{Ativa}}: Deve-se resolver o problema novamente.
			\item[$\diamond$] \textbf{\underline{Não está ativa}}: A desigualdade pode ser eliminada, pois esta não afetará em nada a solução do problema.
			\end{itemize}
	\end{itemize}
\end{frame}

\begin{frame}
	\frametitle{Eliminação de Restrições}
	\begin{columns}
		\begin{column}{5cm}
			\scriptsize
			\begin{table}
				\only<1>
				{
					\begin{tabular}{c}
						\cellcolor{red!50} $ \max z = 600x_1 + 800x_2 $ \\
						\cellcolor{red!50} Sujeito a: \\
						\cellcolor{red!50} $x_1 + x_2 \le 100$ \\
						\cellcolor{red!50} $3x_1 + 2x_2 \le 240$ \\
						\cellcolor{red!50} $x_1 \le 60 $ \\
						\cellcolor{red!50} $x_2 \le 80 $ \\
					    \cellcolor{red!50} $x_1, x_2 \ge 0 $ \\
					\end{tabular}
				}
				\only<2>
				{
					\begin{tabular}{c}
						\cellcolor{red!50} $ \max z = 600x_1 + 800x_2 $ \\
						\cellcolor{red!50} Sujeito a: \\
						\cellcolor{red!50} $x_1 + x_2 \le 100$ {\color{green} $\Leftrightarrow$ Ativa}\\
						\cellcolor{red!50} $3x_1 + 2x_2 \le 240$ {\color{green} $\Leftrightarrow$ \textbf{Não Ativa}}\\
						\cellcolor{red!50} $x_1 \le 60 $ {\color{green} $\Leftrightarrow$ \textbf{Não Ativa}}\\
						\cellcolor{red!50} $x_2 \le 80 $ {\color{green} $\Leftrightarrow$ Ativa}\\
					    \cellcolor{red!50} $x_1, x_2 \ge 0 $ \\
					\end{tabular}
				}
			\end{table}
		\end{column}
		\begin{column}{3cm}
			\centering
			\textbf{Forma Padrão} \\
			\includegraphics[width=3cm,height=0.5cm]{seta.png} \\
			\textbf{Tableau Simplex} \\
		\end{column}
		\begin{column}{4cm}
			\scriptsize
			\begin{table}
				\begin{tabular}{c}
					\cellcolor{red!50} $ \max z = 600x_1 + 800x_2 $ \\
					\cellcolor{red!50} Sujeito a: \\
					\cellcolor{red!50} $x_1 + x_2 {\color{green}+ S_1 =}  100$ \\
					\cellcolor{red!50} $3x_1 + 2x_2 {\color{green}+ S_2 =}  240$ \\
					\cellcolor{red!50} $x_1 {\color{green}+ S_3 =}  60 $  \\
					\cellcolor{red!50} $x_2 {\color{green}+ S_4 =}  80 $ \\
				    \cellcolor{red!50} $x_1, x_2 \ge 0 $ \\
				\end{tabular}
			\end{table}
		\end{column}
	\end{columns}
	
	\begin{table}
		\caption{{\textbf{Tableau Simplex Final Processo Iterativo}}}
		\begin{tabular}{c c c c c c c c c}
			\cellcolor{blue} {\color{white} Base} &
			\cellcolor{blue} {\color{white} $Z$} &
			\cellcolor{blue} {\color{white} $x_1$} &
			\cellcolor{blue} {\color{white} $x_2$} &
			\cellcolor{blue} {\color{white} $S_1$} &
			\cellcolor{blue} {\color{white} $S_2$} &
			\cellcolor{blue} {\color{white} $S_3$} &
			\cellcolor{blue} {\color{white} $S_4$} &
			\cellcolor{blue} {\color{white} $B$} \\
			\cellcolor{blue} {\color{white} $X_1$} &
			\cellcolor{yellow} $0$& 
			\cellcolor{yellow} $1$& 
			\cellcolor{yellow} $0$& 
			\cellcolor{yellow} $1$& 
			\cellcolor{yellow} $0$& 
			\cellcolor{yellow} $0$& 
			\cellcolor{yellow} $-1$&
			\cellcolor{yellow} $20$\\
			\cellcolor{blue} {\color{white} $S_2$} &
			\cellcolor{yellow} $0$& 
			\cellcolor{yellow} $0$& 
			\cellcolor{yellow} $0$& 
			\cellcolor{yellow} $-3$& 
			\cellcolor{yellow} $1$& 
			\cellcolor{yellow} $0$& 
			\cellcolor{yellow} $1$&
			\cellcolor{yellow} $20$\\
			\cellcolor{blue} {\color{white} $S_3$} &
			\cellcolor{yellow} $0$& 
			\cellcolor{yellow} $0$& 
			\cellcolor{yellow} $0$& 
			\cellcolor{yellow} $-2$& 
			\cellcolor{yellow} $0$& 
			\cellcolor{yellow} $1$& 
			\cellcolor{yellow} $1$&
			\cellcolor{yellow} $40$\\
			\cellcolor{blue} {\color{white} $X_2$} &
			\cellcolor{yellow} $0$& 
			\cellcolor{yellow} $0$& 
			\cellcolor{yellow} $1$& 
			\cellcolor{yellow} $0$& 
			\cellcolor{yellow} $0$& 
			\cellcolor{yellow} $0$& 
			\cellcolor{yellow} $1$&
			\cellcolor{yellow} $80$\\
			\only<1>
			{
			\cellcolor{blue} {\color{white} $Z$} &
			\cellcolor{yellow} $1$& 
			\cellcolor{yellow} $0$& 
			\cellcolor{yellow} $0$& 
			\cellcolor{yellow} $600$& 
			\cellcolor{yellow} $0$& 
			\cellcolor{yellow} $0$& 
			\cellcolor{yellow} $200$&
			\cellcolor{yellow} $76000$\\
			}
			\only<2>
			{
			\cellcolor{blue} {\color{white} $Z$} &
			\cellcolor{yellow} $1$& 
			\cellcolor{yellow} $0$& 
			\cellcolor{yellow} $0$& 
			\cellcolor{green} $600$& 
			\cellcolor{green} $0$& 
			\cellcolor{green} $0$& 
			\cellcolor{green} $200$&
			\cellcolor{yellow} $76000$\\
			}
		\end{tabular}
	\end{table}
\end{frame}

\begin{frame}[fragile]
	\frametitle{Eliminação de Restrições}
	\begin{columns}
		\begin{column}{5cm}
			\scriptsize
			\begin{table}
				\begin{tabular}{c}
					\cellcolor{red!50} $ \max z = 600x_1 + 800x_2 $ \\
					\cellcolor{red!50} Sujeito a: \\
					\cellcolor{red!50} $x_1 + x_2 \le 100$ {\color{green} $\Leftrightarrow$ Ativa}\\
					\cellcolor{red!50} $3x_1 + 2x_2 \le 240$ {\color{green} $\Leftrightarrow$ \textbf{Não Ativa}}\\
					\cellcolor{red!50} $x_1 \le 60 $ {\color{green} $\Leftrightarrow$ \textbf{Não Ativa}}\\
					\cellcolor{red!50} $x_2 \le 80 $ {\color{green} $\Leftrightarrow$ Ativa}\\
				    \cellcolor{red!50} $x_1, x_2 \ge 0 $ \\
				\end{tabular}
			\end{table}
		\end{column} \pause
		\begin{column}{3cm}
			\centering
			\includegraphics[width=3cm,height=2cm]{tesoura.jpg} \\
		\end{column}
		\begin{column}{4cm}
			\scriptsize
			\begin{table}
				\begin{tabular}{c}
					\cellcolor{red!50} $ \max z = 600x_1 + 800x_2 $ \\
					\cellcolor{red!50} Sujeito a: \\
					\cellcolor{red!50} $x_1 + x_2 {\color{green}+ S_1 =}  100$ \\
					\cellcolor{red!50} $x_2 {\color{green}+ S_4 =}  80 $ \\
				    \cellcolor{red!50} $x_1, x_2 \ge 0 $ \\
				\end{tabular}
			\end{table}
		\end{column} \pause
	\end{columns}
	\begin{columns}
		\begin{column}{6cm}		
			\centering
			\begin{block}{Programa em Matlab}
				\begin{lstlisting}[basicstyle=\tiny]  
clear all; close all; clc;

c = -[ 600 800];

A = [ 1 1; 0 1]; B = [ 100; 80];
lb = [0 0]; ub = [inf inf];

[x, fval, exitflag] = linprog(c,A,B,[],[],lb,ub)
				\end{lstlisting}			
			\end{block}
		\end{column} \pause
		\begin{column}{6cm}
			\centering
			\includegraphics[width=3cm,height=4cm]{Elimina_Matlab_3.png}
		\end{column}
	\end{columns}	
\end{frame}

\subsection{Variação dos Coeficientes da FOB}
\begin{frame}
	\frametitle{Variação dos Coeficientes da FOB}
	\begin{block}{Exemplo da Fábrica de Sucos}
		\begin{table}
			\begin{tabular}{l l l}
				$ \max z = 5x_1 + 7x_2 + 3x_3 $	& \color{red} $\Rightarrow$ Lucro  \\
				\underline{Sujeito a}: \\
				$2x_1 + 3x_2 + 4x_3 \le 240$ 	& \color{red} $\Rightarrow$ Horas \\
				$2x_1 + 1x_2 + 1x_3 \le 150$ 	& \color{red} $\Rightarrow$ Matéria Prima \\
				$x_1 \le 80$ 					& \color{red} $\Rightarrow$ Produção \\
				$x_1, x_2, x_3 \ge 0$ \\ 
				& & \scriptsize $x_1$ litros de suco maçã\\
				& & \scriptsize $x_2$ litros de suco uva\\
				& & \scriptsize $x_3$ litros de suco limão\\
			\end{tabular}
		\end{table}
	\end{block}
	\begin{columns}
		\only<2->
		{
			\begin{column}{1cm}
				\centering
				\includegraphics[width=1.5cm,height=2cm]{doubt.jpg}
			\end{column}
		}
		\only<2>
		{
			\begin{column}{10cm}
				\Large
				Uma redução do preço original de $x_3$ tem impacto no lucro (Z)?
			\end{column}
		}
		\only<3>
		{
			\begin{column}{10cm}
				\Large
				A partir de que valor de preço, o produto $x_3$ passa a ser mais vantajoso?
			\end{column}
		}
		\only<4>
		{
			\begin{column}{10cm}
				\Large
				Como ficaria minha estratégia de venda com esse novo valor de produto $x_3$?
			\end{column}
		}


	\end{columns}
\end{frame}

\begin{frame}
	\frametitle{Exemplo da Fábrica de Sucos}
	\begin{table}
		\scriptsize
		\caption{{\textbf{Tableau Simplex Final Processo Iterativo}}}
		\begin{tabular}{c c c c c c c c c}
			\cellcolor{blue} {\color{white} Base} &
			\cellcolor{blue} {\color{white} $Z$} &
			\cellcolor{blue} {\color{white} $X_1$} &
			\cellcolor{blue} {\color{red} $X_2$} &
			\cellcolor{blue} {\color{red} $X_3$} &
			\cellcolor{blue} {\color{white} $SLK_2$} &
			\cellcolor{blue} {\color{white} $SLK_3$} &
			\cellcolor{blue} {\color{red} $SLK_4$} &
			\cellcolor{blue} {\color{white} $B$} \\
			\cellcolor{blue} {\color{red} $X_2$} &
			\cellcolor{yellow} $0$& 
			\cellcolor{yellow} $0$& 
			\cellcolor{yellow} $1$& 
			\cellcolor{yellow} $1.5$& 
			\cellcolor{yellow} $0.5$& 
			\cellcolor{yellow} $-0.5$& 
			\cellcolor{yellow} $0$&
			\cellcolor{yellow} $45$\\
			\cellcolor{blue} {\color{red} $X_1$} &
			\cellcolor{yellow} $0$& 
			\cellcolor{yellow} $1$& 
			\cellcolor{yellow} $0$& 
			\cellcolor{yellow} $-0.25$& 
			\cellcolor{yellow} $-0.25$& 
			\cellcolor{yellow} $0.75$& 
			\cellcolor{yellow} $0$&
			\cellcolor{yellow} $52.5$\\
			\cellcolor{blue} {\color{red} $SLK_4$} &
			\cellcolor{yellow} $0$& 
			\cellcolor{yellow} $0$& 
			\cellcolor{yellow} $0$& 
			\cellcolor{yellow} $0.25$& 
			\cellcolor{yellow} $0.25$& 
			\cellcolor{yellow} $-0.75$& 
			\cellcolor{yellow} $1$&
			\cellcolor{yellow} $27.5$\\
			\only<1>
			{
				\cellcolor{blue} {\color{white} $Z$} &
				\cellcolor{yellow} $1$& 
				\cellcolor{yellow} $0$& 
				\cellcolor{yellow} $0$& 
				\cellcolor{yellow} $6.25$& 
				\cellcolor{yellow} $2.25$& 
				\cellcolor{yellow} $0.25$& 
				\cellcolor{yellow} $0$&
				\cellcolor{yellow} $577.5$\\
			}
			\only<4>
			{
				\cellcolor{blue} {\color{white} $Z$} &
				\cellcolor{yellow} $1$& 
				\cellcolor{yellow} $0$& 
				\cellcolor{yellow} $0$& 
				\cellcolor{green} $6.25$& 
				\cellcolor{yellow} $2.25$& 
				\cellcolor{yellow} $0.25$& 
				\cellcolor{yellow} $0$&
				\cellcolor{yellow} $577.5$\\
			}
		\end{tabular}
	\end{table}
	\begin{columns}
		\begin{column}{5cm}
			\begin{table}
				\scriptsize`
				\begin{tabular}{| l | }
					\hline
					\cellcolor{yellow} $ \max Z = 5X_1 + 7X_2 + 3X_3 $ \\
					\cellcolor{yellow} \underline{Sujeito a}: \\
					\cellcolor{yellow} $2X_1 + 3X_2 + 4X_3 + SLK_2 = 240$ \\
					\cellcolor{yellow} $2X_1 + 1X_2 + 1X_3 + SLK_3 150$ \\
					\cellcolor{yellow} $X_1 + SLK_4 = 80$ \\
					\cellcolor{yellow} $X_1, X_2, X_3, SLK_2, SLK_3, SLK_4 \ge 0$ \\ 
					\hline
				\end{tabular}
			\end{table}			
		\end{column}
		\begin{column}{6cm}
			\begin{block}{CASO 1 - Variável Não Básica}
				Quanto se pode variar o coeficiente da variável $X_3$ (VNB) na FOB \textbf{sem alterar o valor da FOB}?
				\only<2->
				{
					\begin{table}
						\scriptsize
						\begin{tabular}{| c |}
							\hline
							\cellcolor{gray} $\max Z = 5X_1 + 7X_2 + (3 + \Delta C_3)X_3 = 577,5$ \\
							\hline
						\end{tabular}
					\end{table}
				}
				\only<3->
				{
					\begin{table}
						\scriptsize
						\begin{tabular}{| c |}
							\hline
							\cellcolor{gray} $ \Delta C_3 = ????$ \\
							\hline
						\end{tabular}
					\end{table}				
				}
			\end{block}
		\end{column}
	\end{columns}	
\end{frame}

\begin{frame}
	\frametitle{Exemplo da Fábrica de Sucos}
	\begin{table}
		\scriptsize
		\caption{{\textbf{Tableau Simplex Final Processo Iterativo}}}
		\begin{tabular}{c c c c c c c c c}
			\cellcolor{blue} {\color{white} Base} &
			\cellcolor{blue} {\color{white} $Z$} &
			\cellcolor{blue} {\color{white} $X_1$} &
			\cellcolor{blue} {\color{red} $X_2$} &
			\cellcolor{blue} {\color{red} $X_3$} &
			\cellcolor{blue} {\color{white} $SLK_2$} &
			\cellcolor{blue} {\color{white} $SLK_3$} &
			\cellcolor{blue} {\color{red} $SLK_4$} &
			\cellcolor{blue} {\color{white} $B$} \\
			\cellcolor{blue} {\color{red} $X_2$} &
			\cellcolor{yellow} $0$& 
			\cellcolor{yellow} $0$& 
			\cellcolor{yellow} $1$& 
			\cellcolor{yellow} $1.5$& 
			\cellcolor{yellow} $0.5$& 
			\cellcolor{yellow} $-0.5$& 
			\cellcolor{yellow} $0$&
			\cellcolor{yellow} $45$\\
			\cellcolor{blue} {\color{red} $X_1$} &
			\cellcolor{yellow} $0$& 
			\cellcolor{yellow} $1$& 
			\cellcolor{yellow} $0$& 
			\cellcolor{yellow} $-0.25$& 
			\cellcolor{yellow} $-0.25$& 
			\cellcolor{yellow} $0.75$& 
			\cellcolor{yellow} $0$&
			\cellcolor{yellow} $52.5$\\
			\cellcolor{blue} {\color{red} $SLK_4$} &
			\cellcolor{yellow} $0$& 
			\cellcolor{yellow} $0$& 
			\cellcolor{yellow} $0$& 
			\cellcolor{yellow} $0.25$& 
			\cellcolor{yellow} $0.25$& 
			\cellcolor{yellow} $-0.75$& 
			\cellcolor{yellow} $1$&
			\cellcolor{yellow} $27.5$\\
			\cellcolor{blue} {\color{white} $Z$} &
			\cellcolor{yellow} $1$& 
			\cellcolor{yellow} $0$& 
			\cellcolor{yellow} $0$& 
			\cellcolor{green} $6.25$& 
			\cellcolor{yellow} $2.25$& 
			\cellcolor{yellow} $0.25$& 
			\cellcolor{yellow} $0$&
			\cellcolor{yellow} $577.5$\\
		\end{tabular}
	\end{table} 
	
	\begin{table}
		\scriptsize
		\caption{\textbf{Expressão da FOB Tableau Ótimo}}
		\begin{tabular}{| c |}
			\hline
			\cellcolor{gray!50} $ Z + 6,25X_3 + 2,25SLK_2 + 0,25SLK_3 = 577,50$ \\
			\hline
		\end{tabular}
	\end{table} \pause

	\begin{table}
		\scriptsize
		\caption{\textbf{Alteração na FOB devido a variação em $X_3$}}
		\begin{tabular}{| c |}
			\hline
			\cellcolor{gray!50} $ \max Z = 5X_1 + 7X_2 + (3+\Delta C_3)X_3$ \\
			\hline
		\end{tabular}
	\end{table} \pause
	
	\begin{table}
		\scriptsize
		\caption{\textbf{Expressão da FOB para entrada no tableau}}
		\begin{tabular}{| c |}
			\hline
			\cellcolor{gray!50} $ \max Z - 5X_1 - 7X_2 - (3+\Delta C_3)X_3$ = 0 \\
			\hline
		\end{tabular}
	\end{table}		
\end{frame}

\begin{frame}
	\frametitle{Exemplo da Fábrica de Sucos}
	\begin{table}
		\scriptsize
		\caption{\textbf{Alteração na Expressão da FOB no tableau ótimo}}
		\begin{tabular}{| c |}
			\hline
			\cellcolor{yellow!50} $ \max Z - 5X_1 - 7X_2 - (3+\Delta C_3)X_3$ = 0 \\
			\hline
		\end{tabular}		
	\end{table}		
	\vspace{2cm}
	O Tableau ótimo (Maximização) permanecerá o mesmo enquanto o coeficiente de $X_3$ for positivo. \pause

	\begin{table}
		\begin{tabular}{| c | c | c |}
			\cline{1-1}
			\cline{3-3}
			\cellcolor{gray!60} $6,25 - \Delta C_3 \ge 0$ & \pause
			\includegraphics[width=1cm,height=0.3cm]{seta_azul_direita.jpg} & 
			\cellcolor{gray!60} $\Delta C_3 \le 6,25$ \\
			\cline{1-1}
			\cline{3-3}
		\end{tabular}
	\end{table}

\end{frame}

\begin{frame}
	\frametitle{Exemplo da Fábrica de Sucos}
	\includegraphics[width=11cm,height=4cm]{Sensib_1.png}
\end{frame}

\begin{frame}
	\frametitle{Exemplo da Fábrica de Sucos}
	\includegraphics[width=11cm,height=8cm]{Sensib_2.png}
\end{frame}

\begin{frame}
	\frametitle{Exemplo da Fábrica de Sucos}
	\includegraphics[width=11cm,height=8cm]{Sensib_3.png}
\end{frame}
	
\begin{frame}
	\frametitle{Exemplo da Fábrica de Sucos}
	\includegraphics[width=11cm,height=8cm]{Sensib_4.png}
\end{frame}

\begin{frame}
	\frametitle{Exemplo da Fábrica de Sucos}
	\begin{table}
		\scriptsize
		\caption{{\textbf{Tableau Simplex Final Processo Iterativo}}}
		\begin{tabular}{c c c c c c c c c}
			\cellcolor{blue} {\color{white} Base} &
			\cellcolor{blue} {\color{white} $Z$} &
			\cellcolor{blue} {\color{white} $X_1$} &
			\cellcolor{blue} {\color{red} $X_2$} &
			\cellcolor{blue} {\color{red} $X_3$} &
			\cellcolor{blue} {\color{white} $SLK_2$} &
			\cellcolor{blue} {\color{white} $SLK_3$} &
			\cellcolor{blue} {\color{red} $SLK_4$} &
			\cellcolor{blue} {\color{white} $B$} \\
			\cellcolor{blue} {\color{red} $X_2$} &
			\cellcolor{yellow} $0$& 
			\cellcolor{yellow} $0$& 
			\cellcolor{yellow} $1$& 
			\cellcolor{yellow} $1.5$& 
			\cellcolor{yellow} $0.5$& 
			\cellcolor{yellow} $-0.5$& 
			\cellcolor{yellow} $0$&
			\cellcolor{yellow} $45$\\
			\cellcolor{blue} {\color{red} $X_1$} &
			\cellcolor{yellow} $0$& 
			\cellcolor{yellow} $1$& 
			\cellcolor{yellow} $0$& 
			\cellcolor{yellow} $-0.25$& 
			\cellcolor{yellow} $-0.25$& 
			\cellcolor{yellow} $0.75$& 
			\cellcolor{yellow} $0$&
			\cellcolor{yellow} $52.5$\\
			\cellcolor{blue} {\color{red} $SLK_4$} &
			\cellcolor{yellow} $0$& 
			\cellcolor{yellow} $0$& 
			\cellcolor{yellow} $0$& 
			\cellcolor{yellow} $0.25$& 
			\cellcolor{yellow} $0.25$& 
			\cellcolor{yellow} $-0.75$& 
			\cellcolor{yellow} $1$&
			\cellcolor{yellow} $27.5$\\
			\only<1>
			{
				\cellcolor{blue} {\color{white} $Z$} &
				\cellcolor{yellow} $1$& 
				\cellcolor{yellow} $0$& 
				\cellcolor{yellow} $0$& 
				\cellcolor{yellow} $6.25$& 
				\cellcolor{yellow} $2.25$& 
				\cellcolor{yellow} $0.25$& 
				\cellcolor{yellow} $0$&
				\cellcolor{yellow} $577.5$\\
			}
			\only<4>
			{
				\cellcolor{blue} {\color{white} $Z$} &
				\cellcolor{yellow} $1$& 
				\cellcolor{green} $-\Delta C_1$& 
				\cellcolor{yellow} $0$& 
				\cellcolor{yellow} $6.25$& 
				\cellcolor{yellow} $2.25$& 
				\cellcolor{yellow} $0.25$& 
				\cellcolor{yellow} $0$&
				\cellcolor{yellow} $577.5$\\
			}
		\end{tabular}
	\end{table}
	\begin{columns}
		\begin{column}{5cm}
			\begin{table}
				\scriptsize`
				\begin{tabular}{| l | }
					\hline
					\cellcolor{yellow} $ \max Z = 5X_1 + 7X_2 + 3X_3 $ \\
					\cellcolor{yellow} \underline{Sujeito a}: \\
					\cellcolor{yellow} $2X_1 + 3X_2 + 4X_3 + SLK_2 = 240$ \\
					\cellcolor{yellow} $2X_1 + 1X_2 + 1X_3 + SLK_3 150$ \\
					\cellcolor{yellow} $X_1 + SLK_4 = 80$ \\
					\cellcolor{yellow} $X_1, X_2, X_3, SLK_2, SLK_3, SLK_4 \ge 0$ \\ 
					\hline
				\end{tabular}
			\end{table}			
		\end{column}
		\begin{column}{6cm}
			\begin{block}{CASO 2 - Variável Básica}
				Quanto se pode variar o coeficiente da variável $X_1$ (VNB) na FOB \textbf{de modo a manter a otimalidade das variáveis do problema}?
				\only<2->
				{
					\begin{table}
						\scriptsize
						\begin{tabular}{| c |}
							\hline
							\cellcolor{gray} $\max Z = (5+\Delta C_1) X_1 + 7X_2 + 3X_3 = 577,5$ \\
							\hline
						\end{tabular}
					\end{table}
				}
				\only<3->
				{
					\begin{table}
						\scriptsize
						\begin{tabular}{| c |}
							\hline
							\cellcolor{gray} $ \Delta C_1 = ????$ \\
							\hline
						\end{tabular}
					\end{table}				
				}
			\end{block}
		\end{column}
	\end{columns}	
\end{frame}

\begin{frame}
	\frametitle{Exemplo da Fábrica de Sucos}
	\begin{table}
		\scriptsize
		\caption{{\textbf{Método Simplex, quadro final com alteração $\Delta C_1$}}}
		\begin{tabular}{c c c c c c c c c}
			\cellcolor{blue} {\color{white} Base} &
			\cellcolor{blue} {\color{white} $Z$} &
			\cellcolor{blue} {\color{white} $X_1$} &
			\cellcolor{blue} {\color{red} $X_2$} &
			\cellcolor{blue} {\color{red} $X_3$} &
			\cellcolor{blue} {\color{white} $SLK_2$} &
			\cellcolor{blue} {\color{white} $SLK_3$} &
			\cellcolor{blue} {\color{red} $SLK_4$} &
			\cellcolor{blue} {\color{white} $B$} \\
			\cellcolor{blue} {\color{red} $X_2$} &
			\cellcolor{yellow} $0$& 
			\cellcolor{yellow} $0$& 
			\cellcolor{yellow} $1$& 
			\cellcolor{yellow} $1.5$& 
			\cellcolor{yellow} $0.5$& 
			\cellcolor{yellow} $-0.5$& 
			\cellcolor{yellow} $0$&
			\cellcolor{yellow} $45$\\
			\cellcolor{blue} {\color{red} $X_1$} &
			\cellcolor{yellow} $0$& 
			\cellcolor{yellow} $1$& 
			\cellcolor{yellow} $0$& 
			\cellcolor{yellow} $-0.25$& 
			\cellcolor{yellow} $-0.25$& 
			\cellcolor{yellow} $0.75$& 
			\cellcolor{yellow} $0$&
			\cellcolor{yellow} $52.5$\\
			\cellcolor{blue} {\color{red} $SLK_4$} &
			\cellcolor{yellow} $0$& 
			\cellcolor{yellow} $0$& 
			\cellcolor{yellow} $0$& 
			\cellcolor{yellow} $0.25$& 
			\cellcolor{yellow} $0.25$& 
			\cellcolor{yellow} $-0.75$& 
			\cellcolor{yellow} $1$&
			\cellcolor{yellow} $27.5$\\
			\cellcolor{blue} {\color{white} $Z$} &
			\cellcolor{yellow} $1$& 
			\cellcolor{green} $-\Delta C_1$& 
			\cellcolor{yellow} $0$& 
			\cellcolor{yellow} $6.25$& 
			\cellcolor{yellow} $2.25$& 
			\cellcolor{yellow} $0.25$& 
			\cellcolor{yellow} $0$&
			\cellcolor{yellow} $577.5$\\
		\end{tabular}
	\end{table}
	Diante da alteração (VB na FOB) deve-se eliminar o coeficiente de $X_1$, uma vez que na FOB só deve haver VNB.
	\begin{mdframed}[backgroundcolor=gray!50]
		\centering
		$\text{Linha}(Z) = \text{Linha}(Z) + \Delta C_1 \cdot \text{Linha}(X_1) $
	\end{mdframed}
	
	\begin{table}
		\scriptsize
		\caption{{\textbf{Linha da FOB com variação $\Delta C_1$}}}
		\begin{tabular}{c c c c c c c c c}
			\cellcolor{blue} {\color{white} Base} &
			\cellcolor{blue} {\color{white} $Z$} &
			\cellcolor{blue} {\color{white} $X_1$} &
			\cellcolor{blue} {\color{red} $X_2$} &
			\cellcolor{blue} {\color{red} $X_3$} &
			\cellcolor{blue} {\color{white} $SLK_2$} &
			\cellcolor{blue} {\color{white} $SLK_3$} &
			\cellcolor{blue} {\color{red} $S_4$} &
			\cellcolor{blue} {\color{white} $B$} \\
			\cellcolor{blue} {\color{white} $Z$} &
			\cellcolor{yellow} $1$ & 
			\cellcolor{yellow} $0$ & 
			\cellcolor{yellow} $0$ & 
			\cellcolor{yellow} $6.25$& 
			\cellcolor{yellow} $2.25$& 
			\cellcolor{yellow} $0.25$& 
			\cellcolor{yellow} $0$&
			\cellcolor{yellow} $577.5$\\
			\cellcolor{blue} & \cellcolor{yellow}& \cellcolor{yellow} & \cellcolor{yellow} & 
			\cellcolor{yellow} $- 0.25 \Delta C_1$& 
			\cellcolor{yellow} $- 0.25 \Delta C_1$& 
			\cellcolor{yellow} $+ 0.75 \Delta C_1$& \cellcolor{yellow} &
			\cellcolor{yellow} $+ 52.5 \Delta C_1$\\
		\end{tabular}
	\end{table}
\end{frame}

\begin{frame}
	\frametitle{Exemplo da Fábrica de Sucos}
	\includegraphics[width=11cm,height=6cm]{Sensib_5.png}
\end{frame}

\begin{frame}
	\frametitle{Exemplo da Fábrica de Sucos}
	\includegraphics[width=11cm,height=6cm]{Sensib_6.png}
\end{frame}

\begin{frame}
	\frametitle{Exemplo da Fábrica de Sucos}
	\includegraphics[width=11cm,height=8cm]{Sensib_7.png}
\end{frame}

\begin{frame}
\Huge{\centerline{Fim}}
\end{frame}

%----------------------------------------------------------------------------------------

\end{document} 